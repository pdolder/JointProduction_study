%%
%%%%%%%%%%%%%%%%%%%%%%%%%%%%%%%%%%%%%%%%
%% Set up 
\documentclass{nature}
\usepackage[margin=1in]{geometry} % Required to make the margins smaller to fit
% more content on each page 
\usepackage{lineno} % add some line nos to aid reading
\usepackage[utf8]{inputenc}
\usepackage{enumitem}
\usepackage[hyphens]{url} % for breaking url's in the bib
\usepackage{amsmath}
\usepackage{graphicx}

% cite is loaded from nature class
\setlist{itemsep=1pt} % This controls spacing between items in the lists
\setlength\parindent{0pt} % Removes all indentation from paragraphs

% figures here	
\graphicspath{figures/}
%%%%%%%%%%%%%%%%%%%%%%%%%%%%%%%%%%%%%%%%

\title{Working title: Spatial separation of catches in highly mixed fisheries}

\author{Paul J. Dolder$^1$ \& James T. Thorson$^2$ \& Cóilín Minto$^1$ \& A.N.
Others}

%%%%%%%%%%%%%%%%%%%%%%%%%%%%%%%%%%%%%%%%%
\begin{document}
\maketitle

\begin{affiliations}
\item Galway-Mayo Institute of Technology (GMIT) 
\item North West Fisheries Science Center, NOAA
\end{affiliations}

%%%%%%%%%%%%%%%%%%%%%%%%%%%%%%%%%%%%%%%%%
\begin{linenumbers}

\begin{abstract} 
Mixed fisheries have resulted in overexploitation of weaker stocks as
over-quota catches continue in fisheries pursuing available quota of healthy
stocks. As EU fisheries management moves to a system where all fish caught are
counted against the quota (the 'Landings obligation'), from 2019 the challenge
will be to maximise catches within the new constraints. Failure to achieve this
will result in lower productivity as quota for healthier stocks remains
uncaught in order to protect the weaker stocks. Consequently decoupling
exploitation of species caught together in mixed fisheries has become an
important goal for fisheries sustainability. A potential mechanism for
decoupling exploitation is spatial targeting (driven by spatial management or
rights-based incentives), but this remains technically challenging due to
complex fishery dynamics, the co-occurrence of species at fine spatial and
temporal scales and a lack of understanding of how this is driven by community
dynamics.  We set out a framework to understand how spatial community and
fishery dynamics interact to determine species composition in catch. We do so
by applying a spatio-temporal joint species distribution model, VAST, to
characterise the highly mixed fisheries of the Celtic Sea where numerous target
and bycatch species are landed as part of a diverse catch. \\

Clear common drivers for spatial distribution patterns emerge for three species
groups commonly targeted (roundfish flatfish and shelf species) and, while
abundance varies from year to year, the same species groups are commonly found
in higher densities together. More subtle differences in distribution can be
found within a species group, which can be used to adjust catch to species
composition, where practical. This indicates common drivers of distribution for
groups of species and highlights the scale of the challenge in separating
catches within the species-groups using spatial management measures, having
important implications for management of the mixed fisheries under the EU
landings obligation.

[304 words]

\end{abstract}

%%%%%%%%%%%%%%%%%%%%%%%%%%%%%%%%%%%%%%%%%
\section*{}

\subsection{Mixed fisheries and the EU landings obligation} 

Recent decades have seen efforts to reduce exploitation rates in fisheries and
rebuild depleted fish populations start to bear fruit \cite{Worm2009}.
Improved management of fisheries has the potential to increase population sizes
and allow increased catches, yet fisheries catch globally remains stagnant
\cite{FAO2016}.  In light of projected increased demand for fish protein
\cite{B??n??2016} there is an important role for well managed fisheries to play
in supporting future food security \cite{Mcclanahan2015} and so there remains a
need to ensure fisheries are managed efficiently to maximise productivity.

A particular challenge in realising increased catches from rebuilt populations
is maximising yields from mixed fisheries \cite{Branch2008, Kuriyama2016,
	Ulrich2016}. This is because in mixed fisheries, the predominant type
of fishery worldwide, several fish species are caught together in the same net
or fishing operation (known as 'technical interaction'). If managed by
individual quotas and catches do not match available species quotas for a
fishing vessel, either the vessel must stop fishing when the first quota is
reached (the 'choke' species) or there is overexploitation of the weaker
species while fishers continue to catch more healthy species and throw back
('discard') the fish for which they have no quota.

The sustainability of European fisheries have been hampered by this 'mixed
fishery problem' for decades with large-scale discarding \cite{Borges2015,
	Uhlmann2014}.  However, a paradigm shift is being introduced under the
EU Common Fisheries Policy (CFP) reform of 2012 through two significant
management changes.  First, by 2019 all fish that are caught are due to be
counted against the respective stock quota; second, by 2020 all fish stocks
must be fished so as to be able to produce their Maximum Sustainable Yield
(MSY)\cite{EuropeanParliamentandCounciloftheEuropeanUnion2013}. The changes are
expected to contribute to attainment of Good Environmental Status (GES) under
the European Marine Strategy Framework Directive (MSFD;
\cite{EuropeanParliament2008}) and move Europe towards an ecosystem based
approach to fisheries management \cite{Garcia2003}. Unless fishers can avoid
catch of unwanted species they will have to stop fishing when reaching their
first restrictive quota. This introduces a potential significant cost to
fishers of under-utilised quota\cite{Hoff2010a, Ulrich2016} and provides a
strong incentive to mitigate such losses \cite{Condie2013, Condie2013a}. The
ability of fishers to align their catch with available quota depends on being
able to exploit target species while avoiding unwanted catch. Methods by which
fishers can alter their fishing patterns include by switching fishing method
(e.g. trawling to netting), changing technical gear characteristics (e.g.
introducing escapement panels in nets), or the timing and location of fishing
activity \cite{Fulton2011b, vanPutten2012a}.

Spatio-temporal management measures (such as time-limited fishery closures)
have been applied in the past to reduce unwanted catch through changes in
spatial fishing patterns with varying degrees of success (e.g.
\cite{Needle2011, Holmes2011, Beare2010, Dinmore2003}) while move-on rules have
also been proposed or implemented to influence catch rates of particular
vulnerable species in order to reduce or eliminate discards
(e.g.\cite{Gardner2008, Dunn2011, Dunn2014a}). However, such measures have
generally been targeted at individual species without considering associations
and interactions among several species. Highly mixed fisheries are complex with
spatial, technological and community interactions combining. The design of
spatio-temporal management measures which aim to allow exploitation of healthy
stocks while protecting weaker stocks requires understanding of these
interactions on a meaningful scale to managers and fishers. Here, we set out a
framework for understanding these complexities. We do this by implementing a
spatio-temporal dimension reduction method and use the results to draw
inference and create a framework to identify trends common among species groups
to describe where spatial measures can contribute to mitigating unwanted
catches in the Celtic Sea mixed fisheries.

[589 words]

\subsection{Framework for analysing spatio-temporal mixed fisheries
	interactions}

We present a framework for analysing how far spatio-temporal avoidance can go
towards mitigating imbalances in quota in mixed fisheries. We use fisheries
independent survey data to characterise the spatio-temporal dynamics of key
components of a fish community by employing a geostatistical Vector
Autoregressive Spatio-temporal model (VAST). VAST includes i) a factor analysis
decomposition to describe trends in spatio-temporal dynamics of the different
species as a function of \textit{n} latent variables\cite{Thorson2015} to
identify community dynamics and drivers common among species groups. This
allows for inference of the distribution and density of poorly sampled species
through association with the modelled factors. In addition, VAST ii) separately
models spatio-temporal encounter probability and catch rates to allow
identification of differences in associations for distribution of the species
groups and densities upon encounter \cite{Thorson2015a}, employs iii) Gaussian
Markov Random Fields (GMRFs) to capture spatial and temporal correlations
within and among species groups for both encounter probability and catch rates
\cite{Thorson2013}, and iv) is set in a mixed modelling framework to allow
estimation of fixed effects to account for systematic differences driving
encounter and catches, such as survey catch rate differences, while integrating
across random effects which capture the spatio-temporal properties of the fish
community.

[200 words]

\subsection{Dynamics of Celtic Sea fisheries}

We use the fisheries in the Celtic Sea as a case study. The Celtic Sea is a
temperate sea where a large number of species make up the commercial catches.
Fisheries are spatially and temporally complex with mixed fisheries undertaken
by several nations using different gear types \cite{Ellis2000, Gerritsen2012}.
(MORE SPECIFIC - details of number of species in typical landings, key
species).

We parametrise our spatio-temporal model using catch data from seven fisheries
independent surveys undertaken over the period 1990 - 2015 (Table S1) and
include nine of the main commercial species: Atlantic cod (\textit{Gadus
	morhua}), Atlantic haddock (\textit{Melanogrammus aeglefinus}),
Atlantic whiting (\textit{Merlangius merlangus}), European Hake
(\textit{Merluccius merluccius}), white-bellied anglerfish (\textit{Lophius
	piscatorius}), black-bellied anglerfish (\textit{Lophius budegassa}),
megrim (\textit{Lepidorhombus whiffiagonis}), European Plaice
(\textit{Pleuronectes platessa}) and Common Sole (\textit{Solea solea}). These
species make up \textgreater 60 \% of landings by towed fishing gears for the
area (average 2011 - 2015, STECF data). Each species was separated into
juvenile and adult size classes based on their legal minimum conservation
reference size (Table S2).

We analyse the data to understand how the different associations among
species-groups (combination of species and size class) and their potential
drivers affect catch compositions in mixed fisheries. We consider how these
have changed over time, and the implications for mixed fisheries in managing
catches of quota species under the EU's landings obligation.

[223 words]

\subsection{Common spatial patterns driving species associations} A spatial
dynamic factor analysis decomposes the dominant spatial patterns driving
differences in encounter probability and abundance. Figure 1 shows the first
three factors for (a) average spatial encounter probability and (b) average
density. The first three factors account of 83.7 \% of the variance in
encounter probability and 69 \% of the variance in density, respectively. A
clear spatial pattern can been seen both for encounter probability and density,
with a positive association with the first factor in the inshore North Easterly
part of the Celtic Sea into the Bristol Channel and Western English Channel,
moving to a negative association offshore in the south-westerly waters. On the
second factor a North / South split can be seen for encounter probability at
approximately the 49$^{\circ}$ N while density is more driven by a positive
association in the deeper westerly waters.  The opposite is evident on the
third factor, with a positive association with the Easterly waters for
encounter probability and negative with the westerly waters, while density is
driven by a North / South split.

The first factor was highly correlated with log(depth) for both encounter
probability (-0.85, CI = -0.88 to -0.81; Figure S1) and density (-0.71, CI =
-0.77 to -0.65; Figure S2). A random forest classification tree assigned 80 \%
of the variance in the first factor for encounter probability to depth and
predominant substrate type, with the majority (86 \%) of the variance explained
by depth. The variance explained by these variables dropped to 25 \% on the
second factor with a more even split between depth and substrate, while
explaining 60 \% of the variance on the third factor.  For density, the
variables explained less of the variance with 62 \%, 35 \%, and 31 \% for each
of the factors, respectively.

It is clear that depth and to a lesser extent substrate are important
predictors for the main driver of similarities and differences in distributions
and abundances for the different species groups. The first factor correlates
strongly with these variables, despite them not explicitly being incorporated
in the model. The utility of these variables as predictors of species
distributions has been identified in other marine species distribution models
\cite{Robinson2011}; the advantage to the approach taken here is that, where
such data is unavailable at appropriate spatial resolution, the spatial factor
analysis can adequately characterise these influences.

[397 words]

\subsection{Changes in spatial patterns over time, but stability in species
	dynamics}

While there are clear spatial patterns in the factors describing average
encounter probability and density, there are inter-annual differences in factor
coefficients which show less structure (Figures S4, S5). While temperature is
often included as a covariate in species distribution models it was found not
to contribute to the variance in the factor coefficients (Figure S6,
correlations for both encounter probability and density $\sim$ 0).

While spatio-temporal factor coefficients did not show common trends, among
species groups there were common dynamics. Figure 2a shows that the same
factors appear to drive spatio-temporal distributions of megrim, anglerfish
species and hake (the deeper water species, species grouping negatively
associated with the second axes) on the one hand and the roundfish and flatfish
on the other. For spatio-temporal density (Figure 2b) cod, haddock and whiting
(the roundfish species) are separated from plaice, sole (the flatfish) and
deeper water species. As such, higher catches the other roundfish species group
would be expected when catching one species group. This suggests that a common
environmental driver is influencing the distributions of the species groups.  

[178 words]

\subsection{Correlations show three distinct species-group associations}
Pearson correlation coefficients for the modelled average spatial encounter
probability (Figure 3a) show clear strong associations between adult and
juvenile size classes for all species (\textgreater 0.75 for all species except
Hake, 0.56).  Among species groups, hierarchical clustering identified three
common groups; Roundfish (cod, haddock, whiting) are found closely associated,
with correlations for adult cod with adult haddock and adult whiting of 0.73
and 0.5 respectively, while adult haddock with adult whiting was 0.63. Flatfish
(plaice and sole) are also strongly correlated with adult plaice and sole
having a coefficient of 0.75.  The final group are principally species found in
the deeper waters (hake, megrim and both anglerfish species) with the megrim
strongly associated with the budegassa anglerfish species (0.88). Negative
relationships were found between plaice and sole and the monkfish species
(-0.27, -0.26 for the adult size class with budegassa adults respectively) and
hake (-0.33, -0.37) indicating spatial separation in distributions.

Correlation coefficients for the average density (Figure 3b) show less
significant relationships then for encounter probability, but still evident are
the strong association among the roundfish with higher catches of cod are
associated with higher catches of haddock (0.58) and whiting (0.47), as well as
the two anglerfish species (0.71 for piscatorius and 0.44 for budegassa) and
hake (0.73). Similarly, plaice and sole are closely associated (0.31) and
higher catches of one would expect to see higher catches of the other, but also
higher catches of some juvenile size classes of roundfish (whiting and haddock)
and anglerfish species. Negative association of juvenile megrim, anglerfish
(budegassa) and hake with adult sole (-0.61, -0.61 and -0.47 respectively),
plaice (-0.36 and -0.35 for megrim and hake only) indicate generally high
abundance of one can predict low abundance of the other successfully.

[[In addition to the average correlation, we also estimate how correlations
change from one year to the next.  The spatial correlations (representing the
average correlations across all years) from the population correlations show
reasonable alignment with the spatio-temporal population correlations
(representing how correlations change from year to year) indicating generally
predictable relationships between species groups from one year to the next
(0.59 (0.52 - 0.66) and 0.47 (0.38 - 0.55) for encounter probability and
density respectively).  However, a linear regression between the spatial
correlations and the spatio-temporal correlations shows high variance
(R\textsuperscript{2} = 0.36 and 0.22 respectively), indicating that the scale
of these relationships do change from one-year to the next. This would have
implications for the predictability of the relationship between catches of one
species group and another. It can also be seen in the spatial factor scores
that there are subtle differences in spatial patterns in factor coefficients
from one year to the next (Figures S5 and S6) indicating changes being driven
by temporally changing environmental factors.]]

[459 words]


\subsection{Subtle differences in distributions may be important to separate
	catches within groups under the landing obligation} 
The analysis shows the interdependence within species groups (roundfish,
flatfish and deeper water species) which has important implications for how
spatial avoidance can be used to support implementation of the EU's landings
obligation. If mixed fisheries are to maximise productivity, decoupling catches
of species between and among the groups will be key.  Methods to do this
include changes to the selectivity characteristics of gear (e.g.
\cite{Santos2016}) and spatio-temporal avoidance. Both are likely to play a
role but the extent to which they can contribute to fisheries sustainability is
unknown. While here we demonstrate the separation of three distinct species
groupings (roundfish, flatfish and deeper water species) separating catches
within groups is likely to be equally as important.  Figure 4 shows the
difference in spatial distribution within a group for each of the species
groupings for a single year (2015). 

Figure 4a indicates that cod had a more North-westerly distribution than
haddock while cod had more westerly distribution than whiting roughly
delineated by the 7$^{\circ}$ W line. Whiting appeared particularly
concentrated in an area between 51 and 52 $^{\circ}$ N and 5 and 7 $^{\circ}$
W, which can be seen by comparing the whiting distribution with both cod
(Figure 4b) and haddock (Figure 4c). For the deeper water species Figures 4d
and 4e indicate that hake are particularly concentrated in two areas compared
to anglerfishes\footnote{two species combined as they are managed as one} and
megrim (though for megrim, a fairly even relative distribution elsewhere is
indicated by the large amount of white space). For anglerfishes and megrim
(Figure 4f), anglerfishes have a more easterly distribution than megrim. For
the flatfish species plaice and sole (Figure 4g), sole appear to be more
concentrated along the coastal areas of Ireland and the UK, while Plaice are
more concentrated in the Southern part of the English Channel along the coast
of France.

These nuanced differences in distribution can have important implications for
fishers seeking to fish in areas which match their quota holdings. Figure 4h
shows the predicted catch distribution from a "typical" Otter trawl gear and
Beam trawl gear fishing at three different locations. As can be seen, both the
gear selectivity and area fished play important contributions to the catch
compositions; in the inshore area (66) plaice and sole are the two main species
in catch reflecting their distribution and abundance, though the Otter trawl
gear catches a greater proportion of plaice than the Beam trawl. The area
between the UK and Ireland (79) has a greater contribution of whiting, haddock,
cod, hake and anglerfishes in the catch with the Otter trawl catching a greater
proportion of the roundfish, haddock, whiting and cod. The offshore area has a
higher contribution of megrim, anglerfishes and hake with the Otter trawl
catching a greater share of hake and the Beam trawl a greater proportion of
megrim. Megrim dominates the catch for both gears in area 216, reflecting its
relative abundance in the area.

[[Figure 5 shows the joint production function for the entire spatial domain,
giving the global production sets for the years 2011 - 2015. It gives the space
in which vessels have to operate where they can change the relative composition
of each species in the catch as a function of changing location fished only.
The convex hull of the space is the flexibility vessels have in order to adapt
to the changing fishing opportunities given the association of species with
each other\cite{Reimer2017}. As can be seen from Figure 5a which shows the
trade-off between cod and haddock for an Otter trawler....

Figure 5b shows the same for plaice and sole for a Beam trawler...
STRUGGLING TO DEFINE WHAT IS 'LOTS' AND WHAT IS 'LITTLE' SPACE
OBJECTIVELY...?]]

[489 words]

\subsection{Application of framework to support implementation landing
	obligation in mixed fisheries}
The framework described and applied here to the Celtic Sea identifies three
species groups that show spatial separation. Within groups spatial separation
is more challenging which has important implications for how the EU's landings
obligation is implemented.


Real time?  Commercial data ?  Hotspots \& Notspots ??

How far can spatial changes in targeting bring you towards achieving the right
balance of catch in mixed fisheries to maintain productivity. FOOD SECURITY!!

\section*{Methods}

\subsection{Model structure:} 

VAST\footnote{software in the R statistical programming language can be found
	here: \url{www.github.com/james-thorson/VAST}} implements a
delta-generalised linear mixed modelling (GLMM) framework that takes account of
spatio-temporal correlations among species-groups through implementation of a
spatial dynamic factor analysis (SDFA).  In the model, spatio-temporal
variation is represented through three-dimensional Gaussian Random Fields while
covariates affecting catchability (to account for differences between fishing
surveys) and density (to account for environmental preferences) can be
incorporated for predictions of presence and density. The following briefly
summarises the key methods implemented in the VAST framework. For full details
of the model the reader is invited directed to Thorson \textit{et al} 2017
\cite{Thorson2017}.

\textbf{\textit{SDFA:}} A spatial dynamic factor analysis incorporates
developments in joint dynamic species models \cite{Thorson2017} to take account
of associations among species / species-groups by modelling response variables
as a multivariate process. This is achieved through implementing a factor
analysis decomposition where common latent trends are estimated so that the
number of common trends (M) is less than the number of species-groups (N)
modelled ($1 \leq N \leq M$). The species group trends are then associated
through a function for each factor that returns a positive or negative
association of one or more species with any location. Log-density of any
species is then be described as a linear combination of factors:
	\begin{equation}
		\theta_{p}(s,t) = \sum_{j=1}^{n_{j}}
		L_{p,j}\psi_{j}(s,t) +\sum_{k=1}^{n_{k}}
		\gamma_{k,p}\chi_{k}(s,t)
	\end{equation}
Where $\theta_{p}(s,t)$ represents log-density for species $p$ at site $s$ at
time $t$, $\psi_{j}$ is the coefficient for factor $j$, $L_{p,j}$ the loading
matrix representing association of species $p$ with factor $j$ and
$\gamma_{k,p}\chi_{k}(s,t)$ the linear effect of covariates at each site and
time \cite{Thorson2016b}. 

The factor analysis can identify community dynamics and where species have
similar spatio-temporal patterns, allowing inference of species distributions
and abundance of poorly sampled species through association with other species
and allows for computation of spatio-temporal correlations among species-groups
\cite{Thorson2016b}.

\textbf{\textit{Estimation of abundances:}} Spatio-temporal encounter
probability and positive catch rates are modelled separately with
spatio-temporal encounter probability modelled using a logit-link linear
predictor;
		\begin{equation}
			logit[p(s_{i},p_{i},t_{i})] = \gamma_{p}(p_{i},t_{i}) +
			\varepsilon_{p}(s_{i},p_{i},t_{i}) + \delta_{p}(p_{i},
			v_{i})
		\end{equation}

and positive catch rates modelling using a gamma- distribution \cite{Thorson2015a}.  
		\begin{equation}
			gamma[r(s_{i},p_{i},t_{i})] = \gamma_{p}(p_{i},t_{i}) +
			\varepsilon_{p}(s_{i},p_{i},t_{i}) + \delta_{p}(p_{i},
			v_{i})
		\end{equation}

With $\gamma_{*}(p_{i},t_{i})$, $\varepsilon_{*}(s_{i},p_{i},t_{i})$ and
$\delta_{*}(p_{i}, v_{i})$ representing an intercept, spatio-temporal variation
and a vessel effect ($v$) respectively for for either probability of encounter,
$p$ or density $r$.

\textbf{\textit{Spatio-temporal variation:}} The spatio-temporal variation is
modelled using Gaussian Markov Random Fields (GMRF) where data is associated to
nearby locations through a Matérn covariance function with the parameters
estimated within the model. Here, the correlation decays smoothly over
space/time the further from the location and includes geometric anisotropy to
reflect the fact that correlations may decline in one direction faster than
another (e.g. moving offshore) \cite{Thorson2013}.  The best fit estimated an
anisotropic covariance where the correlations were stronger in a North-East -
South-West direction, extending approximately 97 km and 140 km before
correlations for encounter probability and density reduced to \textless 10 \%,
respectively (Figure S10).  Incorporating the spatio-temporal correlations
among and within species provides more efficient use of the data as inference
can be made about poorly sampled locations from the covariance structure.

A probability distribution for spatio-temporal variation in both encounter
probability and positive catch rate was specified, $\varepsilon_{*}(s,p,t)$,
with a three-dimensional multivariate normal distribution so that:
	\begin{equation}
		vec[\mathbf{E}_{*}(t)] \sim MVN(0,\mathbf{R}_{*} \otimes
		\mathbf{V}_{{\varepsilon}{*}})
	\end{equation}

Here, $vec[\mathbf{E}_{*}(t)]$ is the stacked columns of the matrices
describing $\varepsilon{*}(s,p,t)$ at every location, species and time,
$\mathbf{R}_{*}$ is a correlation matrix for encounter probability or positive
catch rates among locations and $\mathbf{V}_{*}$ a correlation matrix for
encounter probability or positive catch rate among species (modelled within the
factor analysis). $\otimes$ represents the Kronecker product so that the
correlation among any location and species can be computed \cite{Thorson2017}.
		
\textbf{\textit{Incorporating covariates}} Survey catchability (the relative
efficiency of a gear catching a species-group) was estimated as a fixed effect
in the model, $\delta_{s}(v)$, to account for differences in spatial fishing
patterns and gear characteristics which affect encounter and capture
probability of the sampling gear \cite{Thorson2014}. Parameter estimates
(Figure S11) showed clear differential effects of surveys using otter trawl
gears (more effective for round fish species) and beam trawl gears (more
effective for flatfish species).

No fixed covariates for habitat quality or other predictors of encounter
probability or density were include. While incorporation may improve the
spatial predictive performance \cite{Thorson2017}, it was not found to be the
case here based on model selection with Akaike Information Criterion (AIC) and
Bayesian Information Criterion (BIC).

\textbf{\textit{Parameter estimation}} Parameter estimation was undertaken
through Laplace approximation of the marginal likelihood for fixed effects
while integrating the joint likelihood (which includes the probability of the
random effects) with respect to random effects. This was implemented using
Template Model Builder (TMB; \cite{Kristensen2015}) with computation through
support by the Irish Centre for High End Computing (ICHEC;
\url{https://www.ichec.ie}) facility.  

\subsection{Data}

The model integrates data from seven fisheries independent surveys taking
account of correlations among species-group spatio-temporal distributions and
abundances to predict spatial density estimates consistent with the resolution
of the data. 

The model was been fit to nine species separated into adult and juvenile size
classes (Table S2) to seven survey series (Table S1) in the Celtic Sea bound by
48$^{\circ}$ N to 52 $^{\circ}$ N latitude and 12 $^{\circ}$ W to 2$^{\circ}$ W
longitude (Figure S9) for the years 1990 - 2015 inclusive. 

The following steps were undertaken for data processing: i) data for survey
stations and catches were downloaded from ICES Datras
(\url{www.ices.dk/marine-data/data-portals/Pages/DATRAS.aspx}) or obtained
directly from the Cefas Fishing Survey System (FSS); ii) data were checked and
any tows with missing or erroneously recorded station information (e.g. tow
duration or distance infeasible) removed; iii) swept area for each of the
survey tows was estimated based on fitting a GAM to gear variables so that
Doorspread = s(Depth) + DoorWt + WarpLength + WarpDiameter + SweepLength and a
gear specific correction factor taken from the literature \cite{Piet2009}; iii)
fish lengths were converted to biomass (Kg) through estimating a von
bertalanffy length weight relationship, $Wt = a \cdot L^{b}$, fit to sampled
length and weight of fish obtained in the EVHOE survey and aggregated within
size classes (adult and juvenile). 

The final dataset comprised of estimates of catches (including zeros) for each
station and species-group and estimated swept area for the tow.

[1019 words]

\subsection{Model setup}

The spatial domain was setup to include 250 knots representing the Gaussian
Random Fields. The model was configured to estimate nine factors to describe
the spatial and spatio-temporal encounter probability and density parameters,
with a log-link between the logit encounter probability and assumed gamma
distribution on positive catches.

Three candidate models were identified, i) a base model where the vessel
interaction was a random effect, ii) the base but where the vessel x species
effect was estimated as a fixed covariate, iii) as in ii) but with the addition
of estimating fixed density covariates for both predominant habitat type at a
knot and depth. AIC and BIC model selection favoured the second model (Table
S3). The final model included estimating 130,950 parameters (1,674 fixed and
129,276 random effects).

\subsection{Model validation}

Q-Q plots show good fit between the derived estimates and the data for positive
catch rates and between the predicted and observed encounter probability.
Further, model outputs are consistent with stock-level trends abundances over time
from international assessments, yet also provide detailed insight into species
co-occurrence and the strength of associations in space and time. 

Total words: \\
Abstract:	 	  245  / 150 \\
Intro:   		  577  \\
Outline:		  203 \\
Case study desc:	  217 \\
Results 1:		  397 \\
Results 2:		  178 \\
Results 3:		  291 \\
Discussion:	          573 \\
Conclusions:	              \\
Methods:		 1019 1237 \\
\\
TOTAL:			 3918 \\
Total - abstract         3673 / 3500 \\

Figures: 5 / 6 \\
References: 37 / 50 \\

%%%%%%%%%%%%%%%%%%%%%%%%%%%%%%%%%%%%%%%%%%%%%%%%%%%%%%%%%%%%%%%%%
\end{linenumbers}
\newpage
\Urlmuskip=0mu plus 1mu\relax
\bibliographystyle{naturemag}
\small{\bibliography{JSDM}}

%%%%%%%%%%%%%%%%%%%%%%%%%%%%%%%%%%%%%%%%%%%%%%%%%%%%%%%%%%%%%%%%%%%

%% Here is the endmatter stuff: Supplementary Info, etc.
%% Use \item's to separate, default label is "Acknowledgements"

%\begin{addendum}
% \item Put acknowledgements here.
% \item[Competing Interests] The authors declare that they have no
% competing financial interests.
% \item[Correspondence] Correspondence and requests for materials
% should be addressed to A.B.C.~(email: myaddress@nowhere.edu).
% \end{addendum}

%%%%%%%%%%%%%%%%%%%%%%%%%%%%%%%%%%%%%%%%%%%%%%%%%%%%%%%%%%%%%%%%%%%

%%%

%% Figure 1 - spatial factors
\begin{figure}
\begin{center}
	\includegraphics[width=0.8\linewidth]{"Figure 2 - SpatialFactorLoadingsOmega1Omega2"}
	\label{fig:1}
	\caption{Spatial Factor coefficients for (a) the average spatial encounter
		probability and (b) the average density,  on the first 3
		factors. Red: positive association to the factor, Blue:
		negative association}
\end{center}
\end{figure}


%% Figure 2 - PCA style plots
\begin{figure}
\begin{center}
	\includegraphics[width=\linewidth]{"Figure 3 - PCAstyle_Plots_SpatioTemp"}
	\label{fig:2}
	\caption{Position of each species-group on the first two axes from the
		factor analysis for (a) spatio-temporal encounter probability
		and (b) spatio-temporal density}
\end{center}
\end{figure}

\begin{figure}
\begin{center}
	\includegraphics[width = \linewidth]{"Figure 1 - Omega1Omega2_Correlations_blank"}
	\label{fig:3}
	\caption{Inter-species correlations for (a) spatial encounter
		probability over all years and (b) spatial density.
		Species-groups are clustered into three groups based on a
		hierarchical clustering method with non-significant
		correlations (those where the Confidence Interval spanned zero)
		left blank}
	\end{center}
\end{figure}

\begin{figure}
\begin{center}
	\includegraphics[width = \linewidth]{"Figure 4 - DensityDifferencesFigureswithCC"}
	\label{fig:4}
	\caption{Differences in spatial density for pairs of species and
		expected catch rates for two different gears at three different
	locations in 2015}
\end{center}
\end{figure}

\begin{figure}
\begin{center}
	\includegraphics[width=\linewidth]{"Figure 5 - TechnicalEfficiency"}
	\label{fig:5}
	\caption{Example of technical efficiency space}
\end{center}
\end{figure}




\end{document}


