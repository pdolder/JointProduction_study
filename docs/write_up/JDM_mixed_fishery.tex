%%
%%%%%%%%%%%%%%%%%%%%%%%%%%%%%%%%%%%%%%%%
%% Set up 
\documentclass{nature}
\usepackage[margin=1in]{geometry} % Required to make the margins smaller to fit
% more content on each page 
\usepackage{lineno} % add some line nos to aid reading
\usepackage[utf8]{inputenc}
\usepackage{enumitem}
\usepackage[hyphens]{url} % for breaking url's in the bib
\usepackage{amsmath}
\usepackage{graphicx}

% cite is loaded from nature class
\setlist{itemsep=1pt} % This controls spacing between items in the lists
\setlength\parindent{0pt} % Removes all indentation from paragraphs

% change references to parenthesis (from square brackets)
\renewcommand\citeleft{(}
\renewcommand\citeright{)}

% figures here	
\graphicspath{figures/}
%%%%%%%%%%%%%%%%%%%%%%%%%%%%%%%%%%%%%%%%

\title{Using joint species distribution models to inform mixed fishery
management}

\author{Paul J. Dolder$^1$ \& James T. Thorson$^2$ \& Cóilín Minto$^1$}

%%%%%%%%%%%%%%%%%%%%%%%%%%%%%%%%%%%%%%%%%
\begin{document}
\maketitle

\begin{affiliations}
\item Galway-Mayo Institute of Technology (GMIT) 
\item North West Fisheries Science Center, NOAA
\end{affiliations}

%%%%%%%%%%%%%%%%%%%%%%%%%%%%%%%%%%%%%%%%%
\begin{linenumbers}

\begin{abstract}
Mixed fisheries are where more than one species are caught together in the same
fishing operation. Mixed fishery interactions are the pre-eminent management
issue in fisheries managed by landing quotas, as over-quota catches of weaker
stocks leads to overexploitation when pursuing landings of healthy stocks.
Decoupling exploitation of weaker stocks from healthy stocks is vital if mixed
fisheries are to be managed efficiently, yet remains a challenge due to
co-occurrence at fine spatial and temporal scales. We apply a joint species
distribution model (VAST) to the nine most important demersal fish species
caught in the Celtic Sea fisheries, each separated into juvenile and adult size
classes. The model integrates data from seven fisheries independent surveys,
takes account of correlations among species-group spatio-temporal distributions
and abundances and makes use of habitat and depth covariates to predict spatial
density estimates consistent with the resolution of the data. Model outputs are
consistent with stock-level trends abundances over time from international
assessments, yet also provide detailed insight into species co-occurrence and
the strength of associations in space and time. We use the outputs to draw
inference on the challenges in separating catches of key commercial demersal
fish in moving to the EU landing obligation.
\end{abstract}

%%%%%%%%%%%%%%%%%%%%%%%%%%%%%%%%%%%%%%%%%
\section*{Text}



\section*{Methods}

%%%%%%%%%%%%%%%%%%%%%%%%%%%%%%%%%%%%%%%%%%%%%%%%%%%%%%%%%%%%%%%%%%
\end{linenumbers}
\newpage
\Urlmuskip=0mu plus 1mu\relax
%\bibliographystyle{apalike}
%\small{\bibliography{fleet_dynamics}}

%%%%%%%%%%%%%%%%%%%%%%%%%%%%%%%%%%%%%%%%%%%%%%%%%%%%%%%%%%%%%%%%%%%

%% Here is the endmatter stuff: Supplementary Info, etc.
%% Use \item's to separate, default label is "Acknowledgements"

%\begin{addendum}
% \item Put acknowledgements here.
% \item[Competing Interests] The authors declare that they have no
% competing financial interests.
% \item[Correspondence] Correspondence and requests for materials
% should be addressed to A.B.C.~(email: myaddress@nowhere.edu).
% \end{addendum}

%%%%%%%%%%%%%%%%%%%%%%%%%%%%%%%%%%%%%%%%%%%%%%%%%%%%%%%%%%%%%%%%%%%

\begin{figure}
\begin{center}
	\includegraphics[width = \linewidth]{Omega1Epsilon2_Correlations}
	\label{fig:1}
	\caption{Inter-species correlations for (a) spatial encounter
		probability over all years and (b) Spatio-temporal density.
		Species-groups are clustered into three groups based on a
		hierarchical clustering method with NOTE: Final figure will
		have non-sig correlations blanked}
\end{center}
\end{figure}

\begin{figure}
\begin{center}
	\includegraphics[width = \linewidth]{DensityFigureswithCCRevised2}
	\label{fig:2}
	\caption{Differences in spatial density for pairs of species and
		expected catch rates for two different gears at three different
	locations}
\end{center}
\end{figure}

\begin{figure}
\begin{center}
	\includegraphics[width=\linewidth]{PCAstyle_Plots_Spatial_FishPics_Leg}
	\label{fig:3}
	\caption{Position of each species-group on the first two loading axes from the
	factor analysis}
\end{center}
\end{figure}


\begin{figure}
\begin{center}
	\includegraphics[width=\linewidth]{SpatialFactorLoadingsOmega1}
	\label{fig:4}
	\caption{Spatial Factor loadings}
\end{center}
\end{figure}

\end{document}
