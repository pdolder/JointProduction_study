%%
%%%%%%%%%%%%%%%%%%%%%%%%%%%%%%%%%%%%%%%%
%% Set up 
\documentclass{nature}
\usepackage[margin=1in]{geometry} % Required to make the margins smaller to fit
% more content on each page 
\usepackage{lineno} % add some line nos to aid reading
\usepackage[utf8]{inputenc}
\usepackage{enumitem}
\usepackage[hyphens]{url} % for breaking url's in the bib
\usepackage{amsmath}
\usepackage{graphicx}

% cite is loaded from nature class
\setlist{itemsep=1pt} % This controls spacing between items in the lists
\setlength\parindent{0pt} % Removes all indentation from paragraphs

% figures here	
\graphicspath{figures/}
%%%%%%%%%%%%%%%%%%%%%%%%%%%%%%%%%%%%%%%%

\title{Working title: Spatial separation of catches in highly mixed fisheries}

\author{Paul J. Dolder$^1$ \& James T. Thorson$^2$ \& Cóilín Minto$^1$ \& A.N.
Others}

%%%%%%%%%%%%%%%%%%%%%%%%%%%%%%%%%%%%%%%%%
\begin{document}
\maketitle

\begin{affiliations}
\item Galway-Mayo Institute of Technology (GMIT) 
\item North West Fisheries Science Center, NOAA
\end{affiliations}

%%%%%%%%%%%%%%%%%%%%%%%%%%%%%%%%%%%%%%%%%
\begin{linenumbers}

\begin{abstract} 
Mixed fisheries have resulted in overexploitation of weaker stock as over-quota
catches continue in fisheries pursuing available quota of healthy stocks. As EU
fisheries management moves to a system where all fish caught are counted
against the quota from 2019 (the 'Landings obligation') the challenge will be
to maximise catches within the new constraints. Failing to achieve this will
result in lower productivity as quota for healthier stocks remains uncaught in
order to protect the weaker stocks; therefore decoupling exploitation of
species caught together in mixed fisheries has become an important goal for
fisheries sustainability.  A potential mechanism for decoupling exploitation is
spatial management, but this has remained technically challenging due to
co-occurrence of species at fine spatial and temporal scales and a lack of
understanding of how this is driven by community dynamics.  We sought to
understanding community dynamics in the Celtic Sea, a highly mixed fishery, by
applying a joint species distribution model (VAST) to the nine most important
demersal fish species, each separated into juvenile and adult size classes.  \\

Clear common drivers for spatial distribution patterns emerge for three species
groups (roundfish flatfish and shelf species) and, while abundance varies from
year to year, the same species groups are commonly found in higher densities
together. This indicates common drivers of distribution and the scale of the
challenge in separating catches within the species-groups using spatial
management measures, having important implications for management of the mixed
fisheries under the EU landings obligation.

[245 words]

\end{abstract}

%%%%%%%%%%%%%%%%%%%%%%%%%%%%%%%%%%%%%%%%%
\section*{}

\subsection{Mixed fisheries and the EU landings obligation} 

Recent decades have seen efforts to reduce exploitation rates in fisheries
start to rebuild depleted fish populations \cite{Worm2009}.  Improved
management of fisheries has the potential to increase population sizes and
allow increased catches, yet fisheries catch globally remains stagnant
\cite{FAO2016}.  In light of projected increased demand for fish protein
\cite{B??n??2016} there is an important role for well managed fisheries to play
in supporting future food security \cite{Mcclanahan2015} and so there remains a
need to ensure fisheries are managed efficiently to maximise productivity.

A particular challenge in realising increased catches from rebuilt populations
is maximising yields from mixed fisheries. This is because in mixed fisheries,
the predominant type of fishery worldwide, several fish species are caught
together in the same net or fishing operation. If managed by individual quotas
and catches do not match available species quotas for a fishing vessel, either
the vessel must stop fishing when the first quota is reached (known as the
'choke' species) or there is overexploitation of the weaker species while
fishers continue to catch more healthy species and throw back ('discard') the
fish for which they have no quota.

The sustainability of European fisheries have been hampered by this 'mixed
fishery problem' for decades with large-scale discarding. However, a paradigm
shift is being introduced under the EU Common Fisheries Policy (CFP) reform of
2012 \cite{EuropeanParliamentandCounciloftheEuropeanUnion2013} through two
significant management changes. First, by 2019 all fish that are caught are due
to be counted against the respective stock quota; second, by 2020 all fish
stocks must be fished so as to be able to produce their Maximum Sustainable
Yield (MSY). The changes are expected to contribute to attainment of Good
Environmental Status (GES) under the European Marine Strategy Framework
Directive (MSFD; \cite{EuropeanParliament2008}) and move Europe towards an
ecosystem based approach to fisheries management \cite{Garcia2003}. Unless
fishers can avoid catch of unwanted species they will have to stop fishing when
reaching their first restrictive quota. This introduces a potential significant
cost to fishers of under-utilised quota\cite{Hoff2010a, Ulrich2016} and
provides a strong incentive to mitigate such losses \cite{Condie2013,
	Condie2013a}. The ability of fishers to align their catch with
available quota depends on being able to exploit target species while avoiding
unwanted catch. Methods by which fishers can alter their fishing patterns
include by switching fishing method (e.g. trawling to netting), changing
technical gear characteristics (e.g.  introducing escapement panels in nets),
or the timing and location of fishing activity \cite{Fulton2011b,
	vanPutten2012a}.

Spatio-temporal management measures (such as time-limited fishery closures)
have been applied in the past to reduce unwanted catch with varying degrees of
success (e.g. \cite{Needle2011, Holmes2011, Beare2010, Dinmore2003}) while
move-on rules have also been proposed or implemented to influence catch rates
of particular vulnerable species in order to reduce or eliminate discards
(e.g.\cite{Gardner2008, Dunn2011, Dunn2014a}). However, such measures have
generally been targeted at individual species without considering associations
and interactions among several species. Highly mixed fisheries are complex with
spatial, technological and community interactions. The design of
spatio-temporal management measures which aim to allow exploitation of healthy
stocks while protecting weaker stocks requires understanding of these
interactions in a meaningful way for management. Here, we set out a framework
for understanding these complexities. We do this by reducing the dimensionality
of the problem through implementation of a spatial dynamic factor analysis and
use the framework to identify common spatio-temporal trends to describe where
spatial measures can contribute to mitigating unwanted catches in the Celtic
Sea mixed fisheries.
 

[570 words]

\subsection{Framework for analysing spatio-temporal mixed fisheries
	interactions}

We present a framework for analysing how far spatio-temporal avoidance can go
towards mitigating imbalances in quota in mixed fisheries. We use fisheries
independent survey data to characterise the spatio-temporal dynamics of key
components of a fish community and employ a geostatistical Vector
Autoregressive Spatio-temporal model (VAST). VAST includes i) a factor analysis
decomposition to describe trends in spatio-temporal dynamics of the different
species as a function of \textit{n} latent variables\cite{Thorson2015}. This
allows for inference of the distribution and density of poorly sampled species
through association with the modelled factors and identification of community
dynamics and drivers common among species groups. ii) Separately modelled
spatio-temporal encounter probability and catch rates to allow separate
characterisation of the distribution of the species groups and densities upon
encounter \cite{Thorson2015a}, iii) Gaussian Markov Random Fields (GMRFs) are
used to capture spatial and temporal correlations within and among species for
both encounter probability and catch rates \cite{Thorson2013}, iv) the approach
is set in a mixed modelling framework to allow estimation of fixed effects to
account for systematic differences driving encounter and catches, such as
survey catch rate differences, while integrating across random effects which
capture the spatio-temporal properties of the fish community.

[197 words]

\subsection{Dynamics of Celtic Sea fisheries}

We use the fisheries in the Celtic Sea as a case study. The Celtic Sea is a
temperate sea where a large number of species make up the commercial catches.
Fisheries are spatially and temporally complex with mixed fisheries undertaken
by several nations using different gear types \cite{Ellis2000, Gerritsen2012}.
(MORE SPECIFIC - details of number of species in typical landings, key
species).

We parametrise our spatio-temporal model using catch data from seven fisheries
independent surveys undertaken over the period 1990 - 2015 (Table S1) and
include nine of the main commercial species: Atlantic cod (\textit{Gadus
	morhua}), Atlantic haddock (\textit{Melanogrammus aeglefinus}),
Atlantic whiting (\textit{Merlangius merlangus}), European Hake
(\textit{Merluccius merluccius}), white-bellied anglerfish (\textit{Lophius
	piscatorius}), black-bellied anglerfish (\textit{Lophius budegassa}),
megrim (\textit{Lepidorhombus whiffiagonis}), European Plaice
(\textit{Pleuronectes platessa}) and Common Sole (\textit{Solea solea}). These
species make up \textgreater 60 \% of landings by towed fishing gears for the
area (average 2011 - 2015, STECF data). Each species was separated into
juvenile and adult size classes based on their legal minimum conservation
reference size (Table S2).

We analyse the data to understand how the different associations among
species-groups (combination of species and size class) and their potential
drivers. We consider how these have changed over time, and the implications for
mixed fisheries in managing catches of quota species under the EU's landings
obligation.

[217 words]

\subsection{Common spatial patterns driving species associations}
A spatial dynamic factor analysis decomposes the dominant spatial patterns
driving differences in encounter probability and abundance. Figure 1 shows the
first three factors for (a) average spatial encounter probability and (b)
average density. The first three factors account of 83.4 \% of the variance in
encounter probability and 69.2 \% of the variance in density, respectively. A
clear spatial pattern can been seen both for encounter probability and density,
with a positive association with the first factor in the inshore North Easterly
part of the Celtic Sea in into the Bristol Channel and Western English Channel,
moving to a negative association offshore in the south-westerly waters. On the
second factor a North / South split can be seen for encounter probability while
density is more driven by a positive association in the deeper westerly waters.
The opposite is evident on the third factor, with a positive association with
the Easterly waters for encounter probability and negative with the westerly
waters, while density is driven by a North / South split.

The strength of the loading on the first factor was highly correlated with
log(depth) for both encounter probability (-0.85, CI = -0.88 to -0.81; Figure
S1) and density (-0.71, CI = -0.77 to -0.65; Figure S2). A random forest
classification tree assigned 80 \% of the variance in the first factor for
encounter probability to depth and predominant habitat type, with the majority
(86 \%) of the variance explained by depth. The contribution of these
covariates dropped to 25 \% on the second factor with a more even split between
depth and habitat, while explaining 60 \% of the variance on the third factor.
For density, the covariates explained less of the variance with 62 \%, 35 \%,
and 31 \% for each of the factors, respectively.

It is evident that depth and to a lesser extent habitat are important
predictors of similarities and differences in species distributions and
abundances. However,.... [context, link to other studies]

[316 words]

\subsection{Changes in spatial patterns over time, but stability in species
	dynamics}

TALK ABOUT THE FACT THAT THESE SPATIAL PATTERNS CHANGE FROM YEAR TO YEAR..THEN
TALK ABOUT HOW THE SPECIES FACTOR LOADINGS SHOW COMMON PATTERNS

However, the first two axes of the factor loadings for the spatio-temporal
encounter probability and densities show clear associations among species
group. Figure 2a shows that the same factors appear to drive spatio-temporal
distributions of megrim, anglerfish species and hake (the deeper water species,
species grouping negatively loaded on the second axes) on the one hand and the
roundfish and flatfish on the other. For spatio-temporal density (Figure 2b)
cod, haddock and whiting (the roundfish species) are separated from plaice,
sole (the flatfish) and deeper water species. As such, higher catches the other
roundfish species group would be expected when catching one species group.
[more explanation of significance]

[230 words]

\subsection{Pearson's correlations show three distinct species-group
	associations:} 
Pearson correlation coefficients for the modelled average spatial encounter
probability (Figure 3a) show clear strong associations between adult and
juvenile size classes for all species (\textgreater 0.75 for all species except
Hake, 0.56).  Among species groups, hierarchical clustering identified three
common groups; Roundfish (cod, haddock, whiting) are found closely associated,
with correlations for adult cod with adult haddock and adult whiting of 0.73
and 0.5 respectively, while adult haddock with adult whiting was 0.63. Flatfish
(plaice and sole) are also strongly correlated with adult plaice and sole
having a coefficient of 0.75.  The final group are principally species found
along the shelf edge (hake, megrim and both anglerfish species) with the megrim
strongly associated with the budegassa anglerfish species (0.88). Negative
relationships were found between plaice and sole and the monkfish species
(-0.27, -0.26 for the adult size class with budegassa adults respectively) and
hake (-0.33, -0.37) indicating spatial separation in distributions.

Pearson's correlation coefficients for the average density (Figure 3b) show
less significant relationships then for encounter probability, but still
evident are the strong association among the roundfish with higher catches of
cod are associated with higher catches of haddock (0.58) and whiting (0.47), as
well as the two anglerfish species (0.71 for piscatorius and 0.44 for
budegassa) and hake (0.73). Similarly, plaice and sole are closely associated
(0.31) and higher catches of one would expect to see higher catches of the
other, but also higher catches of some juvenile size classes of roundfish
(whiting and haddock) and anglerfish species. Negative association of juvenile
megrim, anglerfish (budegassa) and hake with adult sole (-0.61, -0.61 and -0.47
respectively), plaice (-0.36 and -0.35 for megrim and hake only) indicate
generally high abundance of one can predict low abundance of the other
successfully.

Pearson's correlation coefficients for spatio-temporal encounter probability
and density show similar though weaker correlations\footnote{need some evidence
	these are weaker!!} to the average values described above (Figures S4).
The average values are correlated with the spatio-temporal values indicating
generally similar relationships (0.59 (0.52 - 0.66) and 0.47 (0.38 - 0.55) for
encounter probability and density respectively) though a linear regression
shows high variance (R\textsuperscript{2} = 0.36 and 0.22 respectively)
indicating that the inter-year variations in the inter-species group
correlations drive some differences in species compositions from one-year to
the next.  This can be seen in the spatial factor loadings which show subtle
differences in spatial patterns from one year to the next (Figures S5 and S6). 

[291 words]


\subsection{Implications for species avoidance in mixed fisheries under the
	landings obligation:}
If fisheries managed by single stock quotas under the EU's landings obligation
are to maintain productivity, decoupling catches of stronger species from weaker
species will be crucial.  Methods to do this include changes to the selectivity
characteristics of gear (refs) and spatio-temporal avoidance. Both are likely
to play a role but the extent to which they can contribute to fisheries
sustainability is unknown. While here we demonstrate the separation of three
distinct species groupings (roundfish, flatfish and deeper water species)
separating catches within groups is likely to be equally as important. Figure 4 shows
the difference in spatial distribution within a group for each of the species
groupings for a single year (2015). 

Figure 4a indicates that cod had a more North-westerly distribution than
haddock while cod had more westerly distribution than whiting roughly
delineated by the 7$^{\circ}$ W line. Whiting appeared particularly
concentrated in an area between 51 and 52 $^{\circ}$ N and 5 and 7 $^{\circ}$
W, which can be seen by comparing the whiting distribution with both cod
(Figure 4b) and haddock (Figure 4c). For the deeper water species Figures 4d and 4e
indicate that hake are particularly concentrated in two areas compared to
anglerfishes\footnote{two species combined as they are managed as one} and
megrim (though for megrim, a fairly even relative distribution elsewhere is
indicated by the large amount of white space). For anglerfishes and megrim
(Figure 4f), anglerfishes have a more easterly distribution than megrim. For
the flatfish species plaice and sole (Figure 4g), sole appear to be more
concentrated along the coastal areas of Ireland and the UK, while Plaice are
more concentrated in the Southern part of the English Channel along the coast
of France.

These nuanced differences in distribution can have important implications for
fishers seeking to fish in areas which match their quota holdings. Figure 4h
shows the predicted catch distribution from a "typical" Otter trawl gear and
Beam trawl gear fishing at three different locations. As can be seen, both the
gear selectivity and area fished play important contributions to the catch
compositions; in the inshore area (66) plaice and sole are the two main species
in catch reflecting their distribution and abundance, though the Otter trawl
gear catches a greater proportion of plaice than the Beam trawl. The area
between the UK and Ireland (79) has a greater contribution of whiting, haddock,
cod, hake and anglerfishes in the catch with the Otter trawl catching a greater
proportion of the roundfish, haddock, whiting and cod. The offshore area has a
higher contribution of megrim, anglerfishes and hake with the Otter trawl
catching a greater share of hake and the Beam trawl a greater proportion of
megrim. Megrim dominates the catch for both gears in area 216, reflecting its
relative abundance in the area.

Figure 5 shows the joint production function for the entire spatial domain,
giving the global production sets for the years 2011 - 2015. It gives the space
in which vessels have to operate where they can change the relative composition
of each species in the catch as a function of changing location fished only.
The convex hull of the space is the flexibility vessels have in order to adapt
to the changing fishing opportunities given the association of species with
each other\cite{Reimer2017}. As can be seen from Figure 5a which shows the
trade-off between cod and haddock for an Otter trawler....

Figure 5b shows the same for plaice and sole for a Beam trawler...
STRUGGLING TO DEFINE WHAT IS 'LOTS' AND WHAT IS 'LITTLE' SPACE OBJECTIVELY...?

[573 words]

\subsection{Implications for mixed fishery management under the EU landing
	obligation}



Real time?  Commercial data ?  Hotspots \& Notspots ??

How far can spatial changes in targeting bring you towards achieving the right
balance of catch in mixed fisheries to maintain productivity. FOOD SECURITY!!

\section*{Methods}



The model integrates data from seven fisheries independent surveys taking
account of correlations among species-group spatio-temporal distributions and
abundances to predict spatial density estimates consistent with the resolution
of the data. 

[[Model outputs are consistent with stock-level trends abundances over time
from international assessments, yet also provide detailed insight into species
co-occurrence and the strength of associations in space and time.  We use the
outputs to draw inference on the challenges in separating catches of key
commercial demersal fish in moving to the EU landing obligation.]]



%%%%%%%%%%%%%%%%%%%%%%%%%%%%%%%%%%%%%%%%%%%%%%%%%%%%%%%%%%%%%%%%%%
\end{linenumbers}
\newpage
\Urlmuskip=0mu plus 1mu\relax
\bibliographystyle{naturemag}
\small{\bibliography{JSDM}}

%%%%%%%%%%%%%%%%%%%%%%%%%%%%%%%%%%%%%%%%%%%%%%%%%%%%%%%%%%%%%%%%%%%

%% Here is the endmatter stuff: Supplementary Info, etc.
%% Use \item's to separate, default label is "Acknowledgements"

%\begin{addendum}
% \item Put acknowledgements here.
% \item[Competing Interests] The authors declare that they have no
% competing financial interests.
% \item[Correspondence] Correspondence and requests for materials
% should be addressed to A.B.C.~(email: myaddress@nowhere.edu).
% \end{addendum}

%%%%%%%%%%%%%%%%%%%%%%%%%%%%%%%%%%%%%%%%%%%%%%%%%%%%%%%%%%%%%%%%%%%

\begin{figure}
\begin{center}
	\includegraphics[width = \linewidth]{Omega1Omega2_Correlations}
	\label{fig:1}
	\caption{Inter-species correlations for (a) spatial encounter
		probability over all years and (b) spatial density.
		Species-groups are clustered into three groups based on a
		hierarchical clustering method with non-significant
		correlations (those where the Confidence Interval spanned zero)
		left blank}
\end{center}
\end{figure}

\begin{figure}
\begin{center}
	\includegraphics[width=\linewidth]{SpatialFactorLoadingsOmega1Omega2}
	\label{fig:2}
	\caption{Spatial Factor loadings for (a) the average spatial encounter
		probability and (b) the average density,  on the first 3
		factors. Red: positive association to the factor, Blue:
		negative association}
\end{center}
\end{figure}

\begin{figure}
\begin{center}
	\includegraphics[width=\linewidth]{PCAstyle_Plots_Spatial_FishPics_Leg}
	\label{fig:3}
	\caption{Position of each species-group on the first two loading axes from the
	factor analysis for (a) spatio-temporal encounter probability and (b)
	spatio-temporal density}
\end{center}
\end{figure}

\begin{figure}
\begin{center}
	\includegraphics[width = \linewidth]{DensityFigureswithCCRevised2}
	\label{fig:4}
	\caption{Differences in spatial density for pairs of species and
		expected catch rates for two different gears at three different
	locations in 2015}
\end{center}
\end{figure}

\begin{figure}
\begin{center}
	\includegraphics[width=\linewidth]{TechnicalEfficiency}
	\label{fig:5}
	\caption{Example of technical efficiency space}
\end{center}
\end{figure}




\end{document}


