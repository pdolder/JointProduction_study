%%
%%%%%%%%%%%%%%%%%%%%%%%%%%%%%%%%%%%%%%%%
%% Set up 
\documentclass{nature}
\usepackage[margin=1in]{geometry} % Required to make the margins smaller to fit
% more content on each page 
\usepackage{lineno} % add some line nos to aid reading
\usepackage[utf8]{inputenc}
\usepackage{enumitem}
\usepackage[hyphens]{url} % for breaking url's in the bib
\usepackage{amsmath}
\usepackage{graphicx}

% cite is loaded from nature class
\setlist{itemsep=1pt} % This controls spacing between items in the lists
\setlength\parindent{0pt} % Removes all indentation from paragraphs

% figures here	
\graphicspath{figures/}
%%%%%%%%%%%%%%%%%%%%%%%%%%%%%%%%%%%%%%%%

\title{Using joint species distribution models to inform mixed fishery
	management OR Managing fisheries through understanding of fish
	community dynamics}

\author{Paul J. Dolder$^1$ \& James T. Thorson$^2$ \& Cóilín Minto$^1$ \& A.N.
Others}

%%%%%%%%%%%%%%%%%%%%%%%%%%%%%%%%%%%%%%%%%
\begin{document}
\maketitle

\begin{affiliations}
\item Galway-Mayo Institute of Technology (GMIT) 
\item North West Fisheries Science Center, NOAA
\end{affiliations}

%%%%%%%%%%%%%%%%%%%%%%%%%%%%%%%%%%%%%%%%%
\begin{linenumbers}

\begin{abstract} 
Mixed fisheries, where more than one species are caught together in the same
fishing operation, are the predominant type of fishery worldwide. Managed in
the EU through single stock landings quotas, mixed fisheries have resulted in
overexploitation of weaker stock as over quota catches continue in fisheries
pursuing catches of healthy stocks. As EU fisheries management moves to systems
where all fish caught are counted against the quota from 2019 (the 'Landings
obligation') this burden is reversed. The risk instead is that mixed fisheries
will have lower productivity as quota for healthier stocks remains uncaught in
order to protect the weaker stocks unless management can decouple exploitation
their exploitation, an objective fast becoming an important goal for fisheries
sustainability.  A potential mechanism for decoupling exploitation is spatial
management, but this has remained technically challenging due to co-occurrence
of species at fine spatial and temporal scales and a lack of understanding of
how this is driven by community dynamics.  We sought to understanding community
dynamics in the Celtic Sea by applying a joint species distribution model
(VAST) to the nine most important demersal fish species, each separated into
juvenile and adult size classes.  \\

Clear common drivers for spatial distribution patterns emerge for three species
groups (roundfish flatfish and shelf species) and, while abundance varies from
year to year, the same species groups are commonly found in higher densities
together. This indicates common drivers of distribution and the scale of the
challenge in separating catches within the species-groups using spatial
management measures, having important implications for management of the mixed
fisheries under the EU landings obligation.

[263 words]

\end{abstract}

%%%%%%%%%%%%%%%%%%%%%%%%%%%%%%%%%%%%%%%%%
\section*{}

\subsection{Mixed fisheries and the EU landings obligation} 

Fisheries management has the goal of maximising food production from capture
fisheries (known as 'maximum sustainable yield (MSY)) while protecting all
components of the ecosystem from over-exploitation. In mixed fisheries, the
predominant type of fishery worldwide, several fish species are caught together
in the same net or fishing operation; if managed by individual quotas and
different species have divergent conservation goals, overexploitation of the
weaker species can result while fishers continue to catch more healthy species
and throw back ('discard') the species for which they have no quota.

The sustainability of European fisheries have been hampered by this 'mixed
fishery problem' for decades, but a paradigm shift is due to be introduced
under the EU Common Fisheries Policy (CFP) reform of 2012
\cite{EuropeanParliamentandCounciloftheEuropeanUnion2013}. Here, two
significant management changes have been introduced. First, all fish stocks
must be fished so as to be able to produce their MSY by 2020; second, all fish
that are caught are due to be counted against the respective stock quota by
2019. The changes are expected to contribute to attainment of Good
Environmental Status (GES) under the European Marine Strategy Framework
Directive (MSFD; \cite{EuropeanParliament2008}) and move towards an Ecosystem
based approach to fisheries management. However, these changes are particularly
challenging for those operating in mixed fisheries, as unless fishers can avoid
catch of unwanted species they will have to stop fishing when reaching their
first restrictive quota. The potential cost to fishers in terms of
under-utilised quota is significant \cite{Hoff2010a, Ulrich2016} and there is
strong incentive to mitigate such losses \cite{Condie2013, Condie2013a}.

The ability of fishers to align their catch with available quota depends on
being able to exploit target species while avoiding unwanted catch. Methods by
which fishers can alter their fishing patterns include by switching fishing
method (e.g. trawling to netting) and changing technical gear characteristics
(e.g. introducing escapement panels in nets), or the timing and location of
fishing activity \cite{Fulton2011b, vanPutten2012a}.

Spatio-temporal management measures (such as time-limited fishery closures)
have been applied in the past to reduce unwanted catch with varying degrees of
success (e.g. \cite{Needle2011, Holmes2011, Beare2010, Dinmore2003} while
move-on rules have also been proposed or implemented to influence catch rates
of particular vulnerable species in order to reduce or eliminate discards
\cite{Gardner2008, Dunn2011, Dunn2014a}. However, such measures have generally
been targeted at individual species without considering associations and
interactions among several species. The design of spatio-temporal management
measures which aim to all exploitation of healthy stocks while supporting
weaker stocks requires understanding of the spatio-temporal population dynamics
of fish communities. 

[420 words]

\subsection{Incorporating spatial modelling advances into a single framework}

We employ a geostatistical Vector Autoregressive Spatio-temporal model (VAST)
which includes i) a factor analysis decomposition to describe trends in
spatio-temporal dynamics of the different species as a function of \textit{n}
latent variables. This allows for inference of the distribution and density of
poorly sampled species through association with the modelled factors and
identification of community dynamics and drivers. ii) Spatio-temporal encounter
probability and catch rates are modelled separately to allow separate
characterisation of distributions and densities, iii) Gaussian Markov Random
Fields (GMRFs) are used to capture spatial and temporal correlations within and
among species for both encounter probability and catch rates, iv) this is set
in a mixed modelling framework to allow estimation of fixed effects such as
gear catch rate differences while integrating across random effects.

[128 words]

\subsection{Dynamics of Celtic Sea fisheries}

We use the fisheries in the Celtic Sea as a case study to understand how fish
community dynamics affect the ability of mixed fisheries to change the
composition of their catch through spatial and spatio-temporal avoidance. The
Celtic Sea is a temperate sea where a large number of species make up the
commercial catches (MORE SPECIFIC).

We parameterise our spatio-temporal model using catch data from seven fisheries
independent surveys undertaken over the period 1990 - 2015 [INSERT REF TO TABLE
WITH SURVEYS AS SUPPL] and include nine of the main commercial species:
Atlantic cod (\textit{Gadus morhua}), Atlantic haddock (\textit{Melanogrammus
	aeglefinus}), Atlantic whiting (\textit{Merlangius merlangus}),
European Hake(\textit{Merluccius merluccius}), white-bellied anglerfish
(\textit{Lophius piscatorius}), black-bellied anglerfish (\textit{Lophius
	budegassa}), megrim (\textit{Lepidorhombus whiffiagonis}), European
Plaice (\textit{Pleuronectes platessa}) and Common Sole (\textit{Solea solea}).
Each species was separated into juvenile and adult size classes based on their
legal minimum conservation reference size (INSERT REF TO TABLE WITH SPP SIZES
AS SUPPL). 

[154 words]

\subsection{Pearson's correlations show three distinct species associations:}
Pearson correlation coefficients for the average spatial encounter probability
\ref{fig:1}(a) show clear strong associations between adult and juvenile size
classes for all species (\textgreater 0.75 for all species except Hake, 0.56).
Among species, hierarchical clustering groups three distinct species groups.
Roundfish (cod, haddock, whiting) are found closely associated with each other:
correlations for adult cod with adult haddock and adult whiting were 0.73 and
0.5 respectively, while adult haddock with adult whiting was 0.63. Flatfish
(plaice and sole) are also strongly correlated with adult plaice and sole
having a coefficient of 0.75. The final group is principally species found
along the shelf edge (hake, megrim and both anglerfish species) with the
megrims more strongly associated with the budegassa anglerfish species (0.88).

Pearsons correlation coefficients for 

\section*{Methods}

The model integrates data from seven fisheries independent surveys taking
account of correlations among species-group spatio-temporal distributions and
abundances to predict spatial density estimates consistent with the resolution
of the data. 

[[Model outputs are consistent with stock-level trends abundances over time
from international assessments, yet also provide detailed insight into species
co-occurrence and the strength of associations in space and time.  We use the
outputs to draw inference on the challenges in separating catches of key
commercial demersal fish in moving to the EU landing obligation.]]



%%%%%%%%%%%%%%%%%%%%%%%%%%%%%%%%%%%%%%%%%%%%%%%%%%%%%%%%%%%%%%%%%%
\end{linenumbers}
\newpage
\Urlmuskip=0mu plus 1mu\relax
\bibliographystyle{naturemag}
\small{\bibliography{JSDM}}

%%%%%%%%%%%%%%%%%%%%%%%%%%%%%%%%%%%%%%%%%%%%%%%%%%%%%%%%%%%%%%%%%%%

%% Here is the endmatter stuff: Supplementary Info, etc.
%% Use \item's to separate, default label is "Acknowledgements"

%\begin{addendum}
% \item Put acknowledgements here.
% \item[Competing Interests] The authors declare that they have no
% competing financial interests.
% \item[Correspondence] Correspondence and requests for materials
% should be addressed to A.B.C.~(email: myaddress@nowhere.edu).
% \end{addendum}

%%%%%%%%%%%%%%%%%%%%%%%%%%%%%%%%%%%%%%%%%%%%%%%%%%%%%%%%%%%%%%%%%%%

\begin{figure}
\begin{center}
	\includegraphics[width = \linewidth]{Omega1Omega2_Correlations}
	\label{fig:1}
	\caption{Inter-species correlations for (a) spatial encounter
		probability over all years and (b) spatial density.
		Species-groups are clustered into three groups based on a
		hierarchical clustering method with NOTE: Final figure will
		have non-sig correlations blanked}
\end{center}
\end{figure}

\begin{figure}
\begin{center}
	\includegraphics[width=\linewidth]{PCAstyle_Plots_Spatial_FishPics_Leg}
	\label{fig:2}
	\caption{Position of each species-group on the first two loading axes from the
	factor analysis for (a) spatio-temporal encounter probability and (b)
	spatio-temporal density}
\end{center}
\end{figure}

\begin{figure}
\begin{center}
	\includegraphics[width=\linewidth]{SpatialFactorLoadingsOmega1}
	\label{fig:3}
	\caption{Spatial Factor loadings for the average spatial encounter
		probability on the first 3 factors}
\end{center}
\end{figure}

\begin{figure}
\begin{center}
	\includegraphics[width = \linewidth]{DensityFigureswithCCRevised2}
	\label{fig:4}
	\caption{Differences in spatial density for pairs of species and
		expected catch rates for two different gears at three different
	locations}
\end{center}
\end{figure}

\begin{figure}
\begin{center}
	\includegraphics[width=\linewidth]{TechnicalEfficiency}
	\label{fig:5}
	\caption{Example of technical efficiency space}
\end{center}
\end{figure}




\end{document}


