%%
%%%%%%%%%%%%%%%%%%%%%%%%%%%%%%%%%%%%%%%%
%% Set up 
\documentclass[fleqn,10pt]{wlscirep}
% more content on each page 
\usepackage{lineno} % add some line nos to aid reading
\usepackage[utf8]{inputenc}
\usepackage{enumitem}
\usepackage{amsmath}
\usepackage{graphicx}

\usepackage{url}
\usepackage[strings]{underscore}
% cite is loaded from nature class
\setlist{itemsep=1pt} % This controls spacing between cites in the lists
\setlength\parindent{0pt} % Removes all indentation from paragraphs
% figures here	
%%%%%%%%%%%%%%%%%%%%%%%%%%%%%%%%%%%%%%%%

\title{Working title: Spatial separation of catches in highly mixed fisheries}

\author[1,2,*]{Paul J. Dolder}
\author[3]{James T. Thorson}
\author[1]{Cóilín Minto}

\affil[1]{Marine and Freshwater Research Centre, Galway-Mayo Institute of
	Technology (GMIT), Dublin Road, Galway, H91 T8NW, Ireland }
\affil[2]{Centre for Environment, Fisheries and Aquaculture Science (Cefas),
	Pakefield Road, Lowestoft, Suffolk, NR33 0HT, UK}
\affil[3]{Fisheries Resource Analysis and Monitoring Division, Northwest Fisheries
	Science Center, National Marine Fisheries Service, NOAA, 2725 Montlake
	Blvd E, Seattle, Washington, 98112, USA}

\affil[*]{paul.dolder@gmit.ie}

%%%%%%%%%%%%%%%%%%%%%%%%%%%%%%%%%%%%%%%%%
%%%%%%%%%%%%%%%%%%%%%%%%%%%%%%%%%%%%%%%%%
\begin{abstract} 
%% based on: https://www.nature.com/nature/authors/gta/2c_Summary_para.pdf
%% Basic introduction 
Mixed fisheries are the dominant type of fishery worldwide.  Overexploitation
in mixed fisheries occurs when catches continue for available quota species
while low quota species are discarded. As EU fisheries management moves to
count all fish caught against quota (the ``landing obligation''), the challenge
is to catch available quota within new constraints, else lose productivity. \\ 

%% More detailed background
A mechanism for decoupling exploitation of species caught together is spatial
targeting, which remains challenging due to complex fishery and population
dynamics. 
%% General problem
How far spatial targeting can go to practically separate species is often
unknown and anecdotal.
%% Summarising the main result
We develop a dimension-reduction framework based on joint species distribution
modelling to understand how spatial community and fishery dynamics interact to
determine species and size composition.\\

%%	 Two or three sentences explaining what the main result
In application to the highly mixed fisheries of the Celtic Sea, clear common
spatial patterns emerge for three distinct assemblages. While distribution
varies interannually, the same species are consistently found in higher
densities together, with more subtle differences within assemblages, where
spatial separation may not be practically possible.\\

%%  general context
We highlight the importance of dimension reduction techniques to focus
management discussion on axes of maximal separation and identify spatiotemporal
modelling as a scientific necessity to address the challenges of managing mixed
fisheries.  
\end{abstract}


%%%%%%%%%%%%%%%%%%%%%%%%%%%%%%%%%%%%%%%%%%%%%%%%%
\begin{document}
\maketitle


%%%%%%%%%%%%%%%%%%%%%%%%%%%%%%%%%%%%%%%%%
\begin{linenumbers}

\section*{Introduction \\}
\subsection*{Mixed fisheries and the EU landing obligation \\}  

Recent efforts to reduce exploitation rates in commercial fisheries have begun
the process of rebuilding depleted fish populations\cite{Worm2009}. Improved
management of fisheries has the potential to increase population sizes and
allow increased sustainable catches, yet fisheries catch globally remains
stagnant\cite{FAO2016}. In light of a projected increase in demand for fish
protein\cite{Bene2016} there is an important role for well managed fisheries in
supporting future food security\cite{Mcclanahan2015} necessitating fisheries
are managed efficiently to maximise productivity.\\

A particular challenge in realising increased catches from rebuilt populations
is maximising yields from mixed fisheries\cite{Branch2008, Kuriyama2016,
	Ulrich2016}. In mixed fisheries, the predominant type of fishery
worldwide, several fish species are caught together in the same net or fishing
operation (known as a ``technical interaction''). If managed by individual
quotas, and catches do not match available stock quotas, either a vessel must
stop fishing when the first quota is reached (the ``choke'' species) or
overexploitation of the weaker species occurs while fishers continue to catch
more healthy species and throw back (``discard'') the fish for which they have
no quota\cite{Batsleer2015}. There is, therefore, a pressing need for
scientific tools, which simplify the complexities of mixed fisheries to help
avoid discarding. \\

Mixed fisheries require specific management approaches to avoid overfishing.
Sustainability of European fisheries has been hampered by the ``mixed fishery
problem'' for decades with large-scale discarding resulting\cite{Borges2015,
	Uhlmann2014}. A paradigm shift is being introduced under the EU Common
Fisheries Policy (CFP) reform of 2012 through two significant management
changes.  First, by 2019 all fish that are caught are due to be counted against
the respective stock quota even if they are discarded; second, by 2020 all fish
stocks must be fished at an exploitation rate corresponding to their Maximum
Sustainable Yield
(MSY)\cite{EuropeanParliamentandCounciloftheEuropeanUnion2013}. The changes are
expected to contribute to attainment of the goal of Good Environmental Status
(GES) under the European Marine Strategy Framework Directive
(MSFD;\cite{EuropeanParliament2008}) and move Europe towards an ecosystem based
approach to fisheries management\cite{Garcia2003}. \\

Conflicts between overall management goals and drivers for individual actors
must be overcome to achieve sustainability. Societal objectives for fisheries
to achieve MSY across ecosystem components are paralleled by individual fishers
goals to maximise utility; whether that be profit, income or the continuance of
traditional practices\cite{Holland2008}. Under the new policy, unless fishers
can avoid catch of unwanted species they will have to stop fishing when
reaching their first restrictive quota. This introduces a potential significant
cost to fishers of under-utilised quota\cite{Hoff2010a, Ulrich2016} and
provides a strong incentive to mitigate such losses\cite{Condie2013,
	Condie2013a}. \\

The ability to align catch with available quota depends on being able to
exploit target species while avoiding unwanted catch. Methods by which fishers
can alter their fishing patterns include by switching fishing method (e.g.
trawling to netting), changing technical gear characteristics (e.g.
introducing escapement panels in nets), or altering the timing and location of
fishing activity\cite{Fulton2011b, vanPutten2012a}. For example, otter trawl
gears are known to have higher catch rates of roundfish due to the higher
headline and wider sweeps, which herd demersal fish into the net. Conversely,
beam trawls employ chain mesh to ``dig'' benthic flatfish species, have higher
catch rates for these species\cite{Fraser2008}. Fishing location choice also
has a significant effect of catch\cite{Gerritsen2012}, something that fishers
routinely consider in their decision making based on their own knowledge. \\

In the past, spatiotemporal management measures (such as time-limited fishery
closures) have been applied to reduce unwanted catch with varying degrees of
success (e.g.\cite{Needle2011, Holmes2011, Beare2010, Dinmore2003}) while
move-on rules have also been proposed or implemented to influence catch rates
of particular vulnerable species in order to reduce or eliminate discards
(e.g.\cite{Gardner2008, Dunn2011, Dunn2014a}).  However, such measures have
generally been targeted at individual species without considering associations
and interactions among several species. Highly mixed fisheries are complex with
spatial, technological and community interactions combining. The design of
spatiotemporal management measures that aim to allow exploitation of high quota
stocks while protecting low quota stocks requires understanding these
interactions at a scale meaningful to managers and fishers. While fisheries
surveys and commercial fishing routinely generate a large amount of
geo-referenced information on numbers and weight of fish caught, integrating
spatiotemporal information from across multiple sources fisheries-dependent and
independent survey data requires an effective framework to reduce and
understand the complexities of the system. \\

Here, our goal is to develop a framework for understanding these complexities.
We do so by 1) implementing a spatiotemporal dimension reduction method that
estimates the likely correlation in catches for multiple species at each
fishing location, 2) using the results to draw inference on the
fishery-community dynamics, 3) creating a framework to identify trends common
among species, and 4) describing the potential for and limitation of spatial
measures to mitigate unwanted catches in highly mixed fisheries.\\

\subsection*{Framework for analysing spatiotemporal mixed fisheries
	interactions \\}

We present a framework for analysing how far spatiotemporal avoidance can
contribute towards mitigating imbalances in quota in mixed fisheries.
Fisheries-independent survey data are used to characterise the spatiotemporal
dynamics of key components of a fish community by employing a geostatistical
Vector Autoregressive Spatiotemporal model (VAST).  Therein, a factor analysis
decomposition was used to describe trends in spatiotemporal dynamics of the
different species as a function of latent variables\cite{Thorson2015}
representing spatial variation (9 factors; termed ``average'' spatial
variation) and spatiotemporal variation (9 factors) for encounter probability
and positive catch rates (termed ``positive density'')
separately\cite{Thorson2015a}.  Resultant factor analyses identify community
dynamics and drivers common among 9 species, each analysed separately for
juveniles and adult stages.  We refer to each combination of species and size
class as a ``species", and present results for the 18 species through
transformation of the loading matrices using PCA rotation. This PCA
rotation is used to visualise a reduced number of orthogonal factors representing
average spatial variation or spatiotemporal variation while explaining the
majority of covariation among catch rates, as well as the association of each
species with these maps. We refer to the association of each species with a
given factor as its ``association with this factor", and the value of each
factor at a given location as its `` `coefficient' at that location". By
describing the species dynamics through underlying  spatiotemporal factors we
can take account of how the factors contribute to affect catches of the species
in mixed fisheries. Gaussian Markov Random Fields (GMRFs) capture spatial and
temporal dependence within and among species for both encounter probability and
positive density\cite{Thorson2013}.  VAST is set in a mixed modelling framework
which allows estimation of fixed effects to account for systematic differences
driving encounter and catches, such as differences in sampling efficiency
(catchability), while random effects capture the spatiotemporal dynamics of the
fish community.\\

\subsection*{Dynamics of Celtic Sea fisheries\\}

The highly mixed demersal fisheries of the Celtic Sea are used as a case study.
The Celtic Sea is a temperate sea where fisheries are spatially and temporally
complex; mixed fisheries are undertaken by several nations using different gear
types\cite{Ellis2000, Gerritsen2012}. Close to 150 species have been identified
in the commercial catches of the Celtic Sea, with approximately 30 species
dominating the catch\cite{Mateo2016}.\\

Our spatiotemporal model is parametrised using catch data from seven
fisheries-independent surveys undertaken in the Celtic Sea over the period 1990
- 2015 (Table S1) and include nine of the main commercial species: Atlantic cod
(\textit{Gadus morhua}), Atlantic haddock (\textit{Melanogrammus aeglefinus}),
Atlantic whiting (\textit{Merlangius merlangus}), European Hake
(\textit{Merluccius merluccius}), white-bellied anglerfish (\textit{Lophius
	piscatorius}), black-bellied anglerfish (\textit{Lophius budegassa}),
megrim (\textit{Lepidorhombus whiffiagonis}), European Plaice
(\textit{Pleuronectes platessa}) and Common Sole (\textit{Solea solea}). These
species comprise over 60 \% of landings by towed fishing gears for the area
(average 2011 - 2015\cite{STECF2017}). Each species was separated into juvenile
and adult size classes based on their legal minimum conservation reference size
(Table S2).\\

The data were analysed to understand how the different associations among
species (combination of species and size class) form distinct assemblages with
common drivers of spatiotemporal distributions, and how these affect catch
compositions for fishers operating in mixed fisheries. We consider how these
have changed over time, and the implications for mixed fisheries in managing
catches of quota species under the EU landing obligation.\\

\section*{Results\\}

Using relatively few factors in a spatial dynamic factor analysis the Celtic
Sea demersal fish community can be partitioned into three species assemblages
(roundfish, flatfish and deeper water species).  Within these assemblages there
are common trends in spatiotemporal distributions in encounter probability and
positive density, which can be partitioned into time invariant (``average
effect") spatial trends and time variant (``spatiotemporal") trends. We show
through presentation of factor coefficients that time invariant trends may be
linked to physical characteristics of the system including depth and
predominant substrate type, while species loadings on to time varying spatial
trends show changes in distribution of species over time to be similar within
an assemblage. We demonstrate how this information can be used to help inform
spatial targeting and avoidance of the different assemblages. More nuanced
differences in spatiotemporal distributions exist within an assemblage
presenting a greater challenge to spatially separate catches, yet we show how
this information may be utilised by mangers and fishers to inform ways to
better match catch to quota in highly mixed fisheries through changes in gear
and location fished.

\subsection*{ Spatial distributions indicate three species assemblages
		\\} 
A spatial dynamic factor analysis was used to decompose the dominant spatial
patterns driving differences in average spatial variation. The first three
factors (after PCA rotation) account for 83.7 \% of the between species
variance in the probability of encountering a species (the ``average encounter
probability") and 69 \% of the explained variance in catch rates on encounter
(``average positive density"). A clear spatial pattern can been seen both for
average encounter probability and average positive density, with a positive
coefficient value associated with the first factor in the inshore north
easterly part of the Celtic Sea into the Bristol Channel and Western English
Channel, moving to a negative coefficient value offshore in the south-westerly
waters (Figure 1). The species loadings show plaice, sole and whiting to be
positively associated with the first factor for average encounter probability
while the other species are negatively associated. For average positive
density, positive associations are also found for haddock and juvenile cod.
This is indicative of a more inshore distribution for these species.\\ 

On the second spatial factor for average encounter probability a north / south
split can be seen at approximately 49$^{\circ}$ N while positive density is
more driven by a positive coefficient in the deeper westerly waters as well as
some inshore areas. Species loadings for the second factor indicate there are
positive associations for juvenile monkfish (\emph{L.  piscatorius}), juvenile
hake, juvenile megrim, plaice and juvenile whiting with average positive
density, which may reflect two different spatial distributions in the more
offshore and in the inshore areas (Figure 1).\\

On the third factor, there is a positive coefficient for the easterly waters
for encounter probability and negative coefficient with the westerly waters.
This splits the roundfish species (cod, haddock and whiting that all have a
positive association with the third factor for average encounter probability)
from the rest of the species (that have a negative association). Positive
density is driven by a north / south split (Figure 1), with positive
coefficient values in the northerly areas. Juvenile monkfish (\emph{L.
	budgessa} and \emph{L.  piscatorius}), cod, juvenile haddock, hake,
adult plaice and whiting are also positively associated with the third factor
towards the north while adult monkfish (\emph{L. budgessa} and \emph{L.
	piscatorius}), adult haddock, megrim, juvenile plaice and sole have
negative loadings reflecting their more southerly distribution (Figure 1).\\

While this exploratory factor analysis models unobserved drivers of
distribution, we considered what might be driving the differences seen in the
spatial factor coefficients and species loadings. The first factor was highly
correlated with log(depth) for both average encounter probability coefficients
(-0.85, CI = -0.88 to -0.81; Figure S1) and average positive density
coefficients (-0.71, CI = -0.77 to -0.65; Figure S2). A random forest
classification tree assigned 80 \% of the variance in the first factor for
average encounter probability to depth and predominant substrate type, with the
majority (86 \%) of the variance explained by depth.  The variance explained by
these variables dropped to 25 \% on the second factor with a more even split
between depth and substrate, while explaining 60 \% of the variance on the
third factor.  For average positive density, the variables explained less of
the variance with 62 \%, 35 \%, and 31 \% for each of the factors,
respectively.\\ 

It is clear that depth and to a lesser extent substrate are important variables
for describing the main driver of similarities and differences in distributions
and abundances for the different species. The first factor correlates strongly
with these variables, despite them not explicitly being incorporated in the
model. While depth and substrate were incorporated as covariates in an
alternative model formulation (see Methods), they were found not to improve
predictions as the random fields adequately captured the influence of these
variables on spatial variation in abundance. The utility of these variables as
predictors of species distributions has been identified in other marine species
distribution models \cite{Robinson2011}. The advantage to the approach taken
here is that, where such data is unavailable at appropriate spatial resolution,
the spatial factor analysis has adequately characterised their influences on
species spatial dynamics.\\

\subsection*{Species assemblages show similar spatiotemporal 
	patterns\\} 
While there are clear spatial patterns in the factor coefficients describing
differences in average  encounter probability and positive density (Figure 1),
the interannual differences in factor coefficients show less structure (Figures
S5, S6). These interannual differences are important as they reflect the
ability of fishers to predict where they can target or avoid species from one
year to the next, without which it may be difficult to balance catches with
available quota and avoid unwanted catch.\\

Spatiotemporal factor coefficients for encounter probability and positive
density did not show the same spatial pattern driving species distributions
from year to year but correlation among species showed clear relationships in
species association with spatiotemporal factor coefficients resulting in the
formation of three different assemblages (Figure 2). The same factors appear
to drive spatiotemporal (interannual changes in) distributions of megrim,
anglerfish species and hake (the deeper water species, forming an assemblage
negatively associated with the second axes of Figure 2) and the roundfish and
flatfish (two assemblages more positively associated with the second axes of
Figure 2a). For spatiotemporal positive density (Figure 2b) cod, haddock and
whiting (the roundfish species) are separated from plaice, sole (the flatfish)
and deeper water assemblage. As such, it can be predicted that higher catches
of a species within a assemblage (e.g. cod in roundfish) would be expected when
catching another species within that assemblage (e.g. whiting in roundfish).
This suggests that one or more common environmental drivers are influencing the
distributions of the assemblages, and that driver differentially affects the
different assemblages. Temperature is often included as a covariate in species
distribution models, but was found not to contribute to the variance in the
first factor coefficients (Figure S6, no correlations found for either
spatiotemporal encounter probability or positive density) and so was not
included as a covariate in the final model.\\

\subsection*{Covariance in spatiotemporal abundance within
		species assemblages\\} 
In order to gain greater insight into the community dynamics we considered how
species covary in space and time through correlations among species. Pearson
correlation coefficients for the modelled average spatial encounter probability
(Figure 3a) show clear strong associations between adult and juvenile size
classes for all species (\textgreater 0.75 for all species except hake, 0.56).
Among species, hierarchical clustering identified the same three common
species-groups as our visual inspection of factor loadings above, with
roundfish (cod, haddock, whiting) closely grouped, with correlations for adult
cod with adult haddock and adult whiting of 0.73 and 0.5 respectively, while
adult haddock with adult whiting was 0.63 (Figure 3a). Flatfish (plaice and
sole) are also strongly correlated with adult plaice and sole having a
coefficient of 0.75. The final group are principally the species found in the
deeper waters (hake, megrim and both anglerfish species) with the megrim
strongly associated with the budegassa anglerfish species (0.88). Negative
relationships were found between plaice and sole, and the monkfish species
(-0.27, -0.26 for the adult size class with budegassa adults respectively) and
hake (-0.33, -0.37) (Figure 3a) indicating spatial separation in distributions
, with the flatfish found more inshore. This underscores the correlations among
species seen in associations of each species with factors, with three distinct
assemblages being confirmed.\\

Correlation coefficients for the average positive density (Figure 3b) show
fewer significant positive or negative relationships among species than for
encounter probability, but still evident are the strong correlation among the
roundfish with higher catches of cod correlated with higher catches of haddock
(0.58) and whiting (0.47), as well as the two anglerfish species (0.71 for
piscatorius and 0.44 for budegassa) and hake (0.73). Similarly, plaice and sole
are closely correlated (0.31) and higher catches of one would expect to see
higher catches of the other, but also higher catches of some juvenile size
classes of roundfish (whiting and haddock) and anglerfish species. Negative
correlation of juvenile megrim, anglerfish (budegassa) and hake with adult sole
(-0.61, -0.61 and -0.47 respectively), plaice (-0.36 and -0.35 for megrim and
hake only) indicate high catches of one can predict low catches of the other
successfully.\\

To understand how stable relationships between catches of pairs of species were
from one year to the next, we regressed the correlation coefficients for the
average spatial correlations between pairs for species \textit{x} and species
\textit{y} across all years (Figure 3) with those of the spatiotemporal
population correlations, representing how correlations between species
\textit{x} and species \textit{y} change from year to year (Figure S9). The
correlations were 0.60 (0.52 - 0.66) and 0.47 (0.38 - 0.55) for encounter
probability and positive density respectively (Figures S9a and S9b). These
indicate generally predictable relationships between species from one year to
the next and suggests that a positive or negative correlation between two
species is likely to persist from one year to the next, and that species are
consistently correlated in hauls.  However, the regressions between the spatial
correlations and the spatiotemporal correlations shows high variance
(R\textsuperscript{2} = 0.36 and 0.22 respectively), indicating that the scale
of these relationships do change from one year to the next. This
unpredictability would have implications for the fishery if, for example,
catches of an unwanted species increased when caught with a target species
above a level expected in the fishery  potentially leading to challenges for
fishers when trying to balance catch with quotas in mixed fisheries. It can  be
seen in the spatial factor maps that there are subtle differences in  patterns
in spatial factor coefficients from one year to the next (Figures S4 and S5),
indicating changes may be driven by temporally changing environmental factors
and species behaviour.\\

\subsection*{ Potential to separate catches within assemblages under the
	landing obligation\\} 
The analysis shows the interdependence within three assemblages of roundfish,
flatfish and deeper water species, where catching one species within the group
indicates a high probability of catching the other species. This has important
implications for how spatial avoidance can be used to support implementation of
the EU's landing obligation. If production from mixed fisheries is to be
maximised, decoupling catches of species between and within the groups will be
key. For example, asking where the maximal separation in the densities of two
coupled species is likely to occur? To address this requirement, we map the
difference in spatial distribution within a species-group for each pair of
species for a single year (2015; Figure 4). \\

Cod had a more north-westerly distribution than haddock and a more westerly
distributed than whiting roughly delineated by the 7$^{\circ}$ W line (Figure
4a). Whiting appeared particularly concentrated in an area between 51 and 52
$^{\circ}$ N and 5 and 7 $^{\circ}$ W, which can be seen by comparing the
whiting distribution with both cod (Figure 4b) and haddock (Figure 4c). For the
deeper water species hake are more densely distributed in two locations around
10 W and 48 N and 12 W and 50 N compared to the anglerfish species
(anglerfishes have been presented together as they are jointly managed under a
single quota) and megrim which were more widely spatially distributed(Figures
4d, and 4e). Megrim has a fairly stable density across the modelled area as
indicated by the large amount of white space in Figures 4e. For anglerfishes
and megrim (Figure 4f), anglerfishes have a more easterly distribution than
megrim.  For the flatfish species plaice and sole (Figure 4g), plaice appear to
be more densely distributed along the coastal areas of Ireland and Britain,
while sole are more densely distributed in the Southern part of the English
Channel along the coast of France.\\

Predicted catch distribution from a ``typical'' otter trawl gear and beam trawl
fishing at three different locations highlights the differences fishing gear
makes on catches (Figure 4h). As can be seen, both the gear selectivity and
location fished play important contributions to the catch compositions; in the
inshore area (location ``A'') plaice and sole are the two main species in the
catch reflecting their distribution and abundance, though the otter trawl gear
catches a greater proportion of plaice to sole than the beam trawl.  The area
between Britain and Ireland (location ``B'') has a greater contribution of
whiting, haddock, cod, hake and anglerfishes in the catch with the otter trawl
catching a greater proportion of the roundfish, haddock, whiting and cod while
the beam trawl catches more anglerfishes and megrim. The offshore area has a
higher contribution of megrim, anglerfishes and hake with the otter trawl
catching a greater share of hake and the beam trawl a greater proportion of
megrim. Megrim dominates the catch for both gears in location ``C'', reflecting
its relative abundance in the area irrespective of the gear deployed.  \\

\section*{Discussion\\}

Our study is framed by the problem of addressing the scientific challenges of
implementing the landing obligation for mixed fisheries. In application to the
Celtic Sea, we have identified spatial separation of three distinct assemblages
(roundfish, flatfish and deeper water species) while showing that only subtle
differences exist in distributions within assemblages. The differences in catch
compositions between gears at the same location (Figure 4h) show that changing
fishing methods can go some way to affecting catch, yet that differences in
catches between locations are likely to be more important. For example, beam
trawls fishing at the inshore locations (e.g. location ``A'' in Figure 4) are
likely to predominately catch plaice and sole, yet switching to the offshore
locations (e.g. location ``C'') would likely yield greater catches of megrim
and anglerfishes.  Such changes in spatial fishing patterns are likely to play
an important role in supporting implementation of the landing obligation.\\

More challenging is within-group spatial separation due to significant overlap
in spatial distributions for the species, driven by common environmental
factors. Subtle changes may yield some benefit in changing catch composition,
yet the outcome is likely to be much more difficult to predict. For example,
subtle differences in the distribution of cod, haddock and whiting can be seen
in Figures 4a-c, showing spatial separation of catches is much more challenging
and likely to need to be supported by other measures such as changes to the
selectivity characteristics of gear\cite{Santos2016}. For example we identified
a spatial overlap of flatfish with juvenile roundfish in our species
correlations (Figure 3); reducing catches of incidental bycatch on the main
target fishing grounds will likely require adaptations to fishing gear to
address bycatch without significant economic impacts on the fishery.\\

A role that science can play in supporting effectiveness of spatiotemporal
avoidance could be to provide probabilistic advice on hotspots for species
occurrence and high species density, which can inform fishing decisions.
Previous modelling studies have shown how spatiotemporal models could improve
predictions of high ratios of bycatch species to target species\cite{Ward2015,
	Cosandey-Godin2015, Breivik2016}, and geostatistical models are well
suited to this as they incorporate spatial dependency while providing for
probabilities to be drawn from posterior distributions of the parameter
estimates. We posit that such advice on ``hot spots" as a supportive measure to
incentivise avoidance of areas of high bycatch risk could be enhanced by
integrating data obtained directly from commercial fishing vessels rapidly
while modelling densities at small time scales (e.g., weekly). Short-term
forecasts of distribution could inform fishing choices while also capturing
seasonal differences in distributions, akin to weather forecasting.  Advice
informed by a model including a seasonal or real-time component could inform
optimal policies for time-area closures, move-on rules or even as informal
information to be utilised by fishers directly without the need for costly
continuous data collection on environmental parameters, but by using the
``vessels-as-laboratories" approach.\\

An important question for the implementation of the EU's landing obligation is
how far spatial avoidance can go to achieving catch balancing in fisheries.
Our model captures differences between location fished for two gear types and
their broad scale effect on catch composition, information crucial for managers
in implementing the landing obligation. It is likely, however, that this
analysis reflects a lower bound on the utility of spatial avoidance as
fine-scale behavioural decisions such as time-of-day, gear configuration and
location choices can also be used to affect catch\cite{Abbott2015,
	Thorson2016}. Results of empirical studies undertaken
elsewhere\cite{Branch2008, Kuriyama2016} suggest limits to the effectiveness of
spatial avoidance. Differences in ability to change catch composition have also
been observed for different fleets; in the North Sea targeting ability was
found to differ between otter and beam trawlers as well as between vessels of
different sizes\cite{Pascoe2007}. \\

Further, under the landing obligation the balance of risk-reward for trip level
fishing decisions about where to fish may change. For example, are fishers
likely to fish in ``safe" areas where its known there are lower catches of the
target species but also decreased risk of encountering bycatch? How do
decisions about level of risk affect the likelihood of overshooting available
quota and potential profit and losses for individual trips? Set in this
context, the parameter estimates could be used to simulate from a distribution
of catches in the fishery at different locations and therefore inform on the
possibility of extreme catch events and potential consequences for overshooting
quotas. This information may be of interest in identifying optimum strategies,
or used in future work to model closure risks for fisheries operating in
different locations and conditions given quota constraints. Such analysis on
risk and decision making is likely to hinge on micro-level decisions by fishers
and such study would be an interesting compliment to broader scale
considerations such as those detailed here. \\

Our framework allows for a quantitative understanding of the broad scale global
production set available to fishers\cite{Reimer2017} and thus the extent to
which they can alter catch compositions while operating in a mixed fishery.
Simulations of spatial effort allocation scenarios based on the production sets
derived from the model estimates could be used as inputs to fisher behavioural
models to allow for identification of the lower bounds of optimum spatial
harvest strategies. Modelling of different spatial strategies at the individual
or fishery level would provide managers with information useful for examining
trade-offs in quota setting by integrating potential for spatial targeting in
changing catch composition, thus provide a scientific contribution to assessing
the ability of technical measures to meet the goal of maximising catches in
mixed fisheries within single stock quota constraints\cite{Ulrich2016}.
Further, the correlations among species could provide information on fisheries
at risk of capturing protected, endangered or threatened species such as
elasmobranches, and allow identification of areas where there are high ratios
of protected to target species.\\

Complex environmental, fishery and community drivers of distribution for groups
of species highlights the scale of the challenge in separating catches within
the assemblages using spatial management measures. This has important
implications for management of the mixed fisheries under the EU landing
obligation. Our analysis identifies where it may be easier to separate catches
of species (among groups) and where it is more challenging (within groups). We
propose that the dimension-reduction framework presented in Figures 1-4
provides a viable route to reducing the complexity of highly mixed fisheries.
This can allow informed management discussion over more traditional anecdotal
knowledge of single-species distribution in space and time.\\

\section*{Methods\\}

\subsection*{Model structure:\\} 

VAST (Software in the R statistical programming language can be found here:
\url{www.github.com/james-thorson/VAST}) implements a delta-generalised linear
mixed modelling (GLMM) framework that takes account of spatiotemporal
correlations among species through implementation of a spatial dynamic factor
analysis (SDFA). Spatial variation is captured through a Gaussian Markov Random
Field, while we model random variation among species and years. Covariates
affecting catchability (to account for differences between fishing surveys) and
density (to account for environmental preferences) can be incorporated for
predictions of presence and positive density. The following briefly summarises
the key methods implemented in the VAST framework. For full details see Thorson
\textit{et al} 2017\cite{Thorson2017}.\\

\textbf{\textit{SDFA:}} A spatial dynamic factor analysis incorporates advances
in joint dynamic species models\cite{Thorson2017} to take account of
associations among species by modelling response variables as a multivariate
process. This is achieved through implementing a factor analysis decomposition
where common latent trends are estimated so that the number of common trends is
less than the number of species modelled. The factor coefficients are then
associated through loadings for each factor that return a positive or negative
association of one or more species with any location. Log-density of any
species is then be described as a linear combination of factors and loadings:
\begin{equation} \theta_{c}(s,t) = \sum_{j=1}^{n_{j}} L_{c,j}\psi_{j}(s,t)
	+\sum_{k=1}^{n_{k}} \gamma_{k,c}\chi_{k}(s,t) \end{equation} Where
$\theta_{c}(s,t)$ represents log-density for species $c$ at site $s$ at time
$t$, $\psi_{j}$ is the coefficient for factor $j$, $L_{c,j}$ the loading matrix
representing association of species $c$ with factor $j$ and
$\gamma_{k,c}\chi_{k}(s,t)$ the linear effect of covariates at each site and
time\cite{Thorson2016b}. \\

The factor analysis can summarize community dynamics and identify which species
and life-stages have similar spatiotemporal patterns. This allows inference
regarding species distributions and abundance of poorly sampled species through
association with other species, and also provides estimates of spatiotemporal
correlations among species\cite{Thorson2016b}.\\

\textbf{\textit{Estimation of abundances:}} Spatiotemporal encounter
probability and positive catch rates are modelled separately with
spatiotemporal encounter probability modelled using a logit-link linear
predictor;
		\begin{equation}
			\begin{split}
			logit[p(s_{i},c_{i},t_{i})] =	\beta_{p}(c_{i},t_{i}) +
			& \sum\limits_{f=1}^{n_{\omega}} L_{\omega}(c_{i},f)
			\omega_{p}(s_{i},f) + \sum\limits_{f=1}^{n_{\varepsilon}}
			L_{\varepsilon}(c_{i},f) \varepsilon_{p}(s_{i},f,t_{i}) + \\ 
			& \sum\limits_{v=1}^{n_{v}}\delta_{p}(v)Q_{p}(c_{i}, v_{i})
		\end{split}
		\end{equation}

and positive catch rates modelling using a gamma- distribution\cite{Thorson2015a}. 
		\begin{equation}
			\begin{split}
			log[r(s_{i},c_{i},t_{i})] = \beta_{r}(c_{i},t_{i}) +
			& \sum\limits_{f=1}^{n_{\omega}} L_{\omega}(c_{i},f)
			\omega_{r}(s_{i},f) +\sum\limits_{f=1}^{n_{\varepsilon}} 
			L_{\varepsilon}(c_{i},f) \varepsilon_{r}(s_{i},f,t_{i}) + \\
			& \sum\limits_{v=1}^{n_{v}}\delta_{r}(v) Q_{r}(c_{i}, v_{i})
			\end{split}
		\end{equation}

where $p(s_{i}, c_{i}, t_{i})$ is the predictor for encounter probability for
observation $i$, at location $s$ for species $c$ and time $t$ and $r(s_{i},
c_{i}, t_{i})$ is similarly the predictor for the positive density.
$\beta_{*}(c_{i},t_{i})$ is the intercept, $\omega_{*}(s_{i},c_{i})$ the
spatial variation at location $s$ for factor $f$, with $L_{\omega}(c_{i},f)$
the loading matrix for spatial covariation among species.
$\varepsilon_{*}(s_{i},c_{i},t_{i})$ is the linear predictor for spatiotemporal
variation, with $L_{\varepsilon}(c_{i}, f)$ the loading matrix for
spatiotemporal covariance among species and $\delta_{*}(c_{i}, v_{i})$ the
contribution of catchability covariates for the linear predictor with
$Q_{c_{i}, v_{i}}$ the catchability covariates for species $c$ and vessel $v$;
$*$ can be either $p$ for probability of encounter or $r$ for positive
density.\\

The Delta-Gamma formulation is then:
\begin{equation}
	\begin{split}
	& Pr(C = 0) = 1 - p \\
	& Pr(C = c | c > 0) = p \cdot \frac{\lambda^{k}c^{k-1} \cdot exp(-\lambda c)}{\Gamma_{k}}
	\end{split}
\end{equation}

for the probability $p$ of a non-zero catch $C$ given a gamma distribution for
for the positive catch with a rate parameter $\lambda$ and shape parameter
$k$.\\

\textbf{\textit{Spatiotemporal variation:}} The spatiotemporal variation is
modelled using Gaussian Markov Random Fields (GMRF) where observations are
correlated in space through a Matérn covariance function with the parameters
estimated within the model. Here, the correlation decays smoothly over space
the further from the location and includes geometric anisotropy to reflect the
fact that correlations may decline in one direction faster than another (e.g.
moving offshore)\cite{Thorson2013}.  The best fit estimated an anisotropic
covariance where the correlations were stronger in a north-east - south-west
direction, extending approximately 97 km and 140 km before correlations for
encounter probability and positive density reduced to \textless 10 \%,
respectively (Figure S10).  Incorporating the spatiotemporal correlations among
and species provides more efficient use of the data as inference can be made
about poorly sampled locations from the covariance structure.\\

A probability distribution for spatiotemporal variation in both encounter
probability and positive catch rate was specified, $\varepsilon_{*}(s,p,t)$,
with a three-dimensional multivariate normal distribution so that:
	\begin{equation}
		vec[\mathbf{E}_{*}(t)] \sim MVN(0,\mathbf{R}_{*} \otimes
		\mathbf{V}_{{\varepsilon}{*}})
	\end{equation}

Here, $vec[\mathbf{E}_{*}(t)]$ is the stacked columns of the matrices
describing $\varepsilon{*}(s,p,t)$ at every location, species and time,
$\mathbf{R}_{*}$ is a correlation matrix for encounter probability or positive
catch rates among locations and $\mathbf{V}_{*}$ a covariance matrix for
encounter probability or positive catch rate among species (modelled within the
factor analysis). $\otimes$ represents the Kronecker product so that the
correlation among any location and species can be computed\cite{Thorson2017}.\\
		
\textbf{\textit{Incorporating covariates}} Survey catchability (the relative
efficiency of a gear catching a species) was estimated as a fixed effect in the
model, $\delta_{s}(v)$, to account for differences in spatial fishing patterns
and gear characteristics, which affect encounter and capture probability of the
sampling gear\cite{Thorson2014}. Parameter estimates (Figure S11) showed clear
differential effects of surveys using otter trawl gears (more effective for
round fish species) and beam trawl gears (more effective for flatfish
species).\\

No fixed covariates for habitat quality or other predictors of encounter
probability or positive density were included. While incorporation may improve
the spatial predictive performance\cite{Thorson2017}, it was not found to be
the case here based on model selection with Akaike Information Criterion (AIC)
and Bayesian Information Criterion (BIC).\\

\textbf{\textit{Parameter estimation}} Parameter estimation was undertaken
through Laplace approximation of the marginal likelihood for fixed effects
while integrating the joint likelihood (which includes the probability of the
random effects) with respect to random effects. This was implemented using
Template Model Builder (TMB;\cite{Kristensen2015}) with computation through
support by the Irish Centre for High End Computing (ICHEC;
\url{https://www.ichec.ie}) facility.  \\

\subsection*{Data\\}

The model integrates data from seven fisheries-independent surveys taking
account of correlations among species spatiotemporal distributions and
abundances to predict spatial density estimates consistent with the resolution
of the data. \\

The model was fitted to nine species separated into adult and juvenile size
classes (Table S2) to seven survey series (Table S1) in the Celtic Sea bound by
48$^{\circ}$ N to 52 $^{\circ}$ N latitude and 12 $^{\circ}$ W to 2$^{\circ}$ W
longitude (Figure S8) for the years 1990 - 2015 inclusive. \\

The following steps were undertaken for data processing: i) data for survey
stations and catches were downloaded from ICES Datras
(\url{www.ices.dk/marine-data/data-portals/Pages/DATRAS.aspx}) or obtained
directly from the Cefas Fishing Survey System (FSS); ii) data were checked and
any tows with missing or erroneously recorded station information (e.g. tow
duration or distance infeasible) removed; iii) swept area for each of the
survey tows was estimated based on fitting a GAM to gear variables so that
Doorspread = s(Depth) + DoorWt + WarpLength + WarpDiameter + SweepLength and a
gear specific correction factor taken from the literature\cite{Piet2009}; iii)
fish lengths were converted to biomass (Kg) through estimating a von
bertalanffy length weight relationship, $Wt = a \cdot L^{b}$, fit to sampled
length and weight of fish obtained in the EVHOE survey and aggregated within
size classes (adult and juvenile). Details on the downloading and processing of
the data are available in Rmarkdown format (code and steps combined) as
supplementary material. \\

The final dataset comprised of estimates of catches (including zeros) for each
station and species and estimated swept area for the tow.\\

\subsection*{Data availability\\}

Data used to fit the model is available via the ICES Datras data portal
(\url{http://www.ices.dk/marine-data/data-portals/Pages/DATRAS.aspx}) for two
surveys and on request to the author for the remaining five surveys.\\

\subsection*{Model setup\\}

The spatial domain was set up to include 250 knots representing the Gaussian
Random Fields. The model was configured to estimate nine factors each to describe
the spatial and spatiotemporal encounter probability and positive density
parameters, with a logit-link for the linear predictor for encounter
probability and log-link for the linear predictor for positive density, with an
assumed gamma distribution.\\

Three candidate models were identified, i) a base model where the vessel
interaction was a random effect, ii) the base but where the vessel x species
effect was estimated as a fixed covariate, iii) with vessel x species effect
estimated, but with the addition of estimating fixed density covariates for
both predominant habitat type at a knot and depth. AIC and BIC model selection
favoured the second model (Table S3). The final model included estimating 1,674
fixed parameters and predicting 129,276 random effect values.\\

\subsection*{Model validation\\}

Q-Q plots show good fit between the derived estimates and the data for positive
catch rates and between the predicted and observed encounter probability (S12,
S13).  Further, model outputs are consistent with stock-level trends abundances
over time from international assessments (S14), yet also provide detailed
insight into species co-occurrence and the strength of associations in space
and time. \\

%%%%%%%%%%%%%%%%%%%%%%%%%%%%%%%%%%%%%%%%%%%%%%%%%%%%%%%%%%%%%%%%%
\end{linenumbers}
\newpage
\Urlmuskip=0mu plus 1mu\relax
%\bibliographystyle{naturemag}
\bibliography{JSDM}

%%%%%%%%%%%%%%%%%%%%%%%%%%%%%%%%%%%%%%%%%%%%%%%%%%%%%%%%%%%%%%%%%%%

\newpage

%% Here is the endmatter stuff: Supplementary Info, etc.
%% Use \item's to separate, default label is "Acknowledgements"


\section*{Acknowledgements} 
Paul J Dolder acknowledges funding support from the MARES joint doctoral
research programme (MARES\_14\_15) and Cefas seedcorn (DP227AC) and logistical
support, desk space and enlightening discussions with Trevor Branch, Peter
Kuriyama, Cole Monnahan and John Trochta.\\
	 
The authors gratefully acknowledge the use of computing facilities at the Irish
Centre for High-End Computing (ICHEC; \url{https://www.ichec.ie}) and the
hard-work of many scientists and crew in collecting and storing data through
the scientific surveys used in this study without which it would not have been
possible.  \\

The manuscript benefited from discussions with D. Stokes, C.  Lordan, C. Moore
and H. Gerritsen (Marine Institute, Ireland), L.  Readdy, C.  Darby, I. Holmes,
S. Shaw and T. Earl (Cefas). We thank J.Hastie, M.McClure, R. Fonner and an
anonymous reviewer for comments that improved the manuscript and Lisa Readdy
for provision of the Cefas datasets.

\section*{Author contributions}
P.J.D., C.M and J.T.T. designed the study. P.J.D. conducted the analysis. All
authors contributed to writing the manuscript.  


\section*{Competing Interests}
The authors declare that they have no competing interests or other interests
that might be perceived to influence the results and/or discussion reported in
this paper.

\section*{Correspondence}
Correspondence and requests for materials should be addressed to Paul Dolder
(email: paul.dolder@gmit.ie).

%%%%%%%%%%%%%%%%%%%%%%%%%%%%%%%%%%%%%%%%%%%%%%%%%%%%%%%%%%%%%%%%%%%

%%%

\newpage

Figure 1: Factor values for the first three factors for (a) Average encounter
probability and (b) Average positive density for the species (outer figures)
and spatially (inner figures).  Red: positive association to the factor, Blue:
negative association.\\ 


Figure 2: Position of each species on the first two axes from the factor
analysis for (a) spatiotemporal encounter probability and (b) spatiotemporal
positive density. Fish images from The Fisherman/Shutterstock.com and Richard
Griffin/Shutterstock.com.\\
	
Figure 3: Inter-species correlations for (a) spatial encounter probability over
all years and (b) spatial positive density.  Species are clustered into three
groups based on a hierarchical clustering method with non-significant
correlations (the Confidence Interval [$\pm$ 1.96 * SEs] spanned zero) left
blank.\\

Figure 4: Differences in the standardised spatial density for pairs of species
and expected catch rates for two different gears at three different locations
in 2015.\\


\end{document}


