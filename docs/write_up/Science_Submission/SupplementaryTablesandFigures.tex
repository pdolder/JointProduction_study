%%%%%%%%%%%%%%%%%%%%%%%%%%%%%%%%%%%%%%%%
%% Set up 
\documentclass[12pt]{article}
\usepackage[margin=1in]{geometry} % Required to make the margins smaller to fit
% more content on each page 
\usepackage{lineno} % add some line nos to aid reading
\usepackage[utf8]{inputenc}
\usepackage{enumitem}
\usepackage[hyphens]{url} % for breaking url's in the bib
\usepackage{amsmath}
\usepackage{graphicx}
\usepackage{subcaption}
\usepackage{scicite}
\usepackage{times}
\usepackage[final]{changes}
\usepackage{xpatch}

%% For editing
\definechangesauthor[color = red]{coilin}
\definechangesauthor[color = blue]{Paul}
\definechangesauthor[color = purple]{brutus}

\newcounter{bibstartvals}
\setcounter{bibstartvals}{26}

\xpatchcmd{\thebibliography}{%
	\usecounter{enumiv}%
}{%
	\usecounter{enumiv}%
	\setcounter{enumiv}{\value{bibstartvals}}%
}{}{}

% figures here	
\graphicspath{'../figures/'}

% The following parameters seem to provide a reasonable page setup.
\topmargin 0.0cm
\oddsidemargin 0.2cm
\textwidth 16cm 
\textheight 21cm
\footskip 1.0cm

%% To allow supplementary table and figure numbering
%% http://bytesizebio.net/2013/03/11/adding-supplementary-tables-and-figures-in-latex/

\newcommand{\beginsupplement}{%
		        \setcounter{table}{0}
		        \renewcommand{\thetable}{S\arabic{table}}%
		        \setcounter{figure}{0}
		        \renewcommand{\thefigure}{S\arabic{figure}}%
			     }

%%  Title
\title{Supplementary material:  Spatial separation of catches in highly mixed fisheries}

\author
{Paul J. Dolder,$^{1,2\ast}$ James T. Thorson,$^3$ \& Cóilín Minto$^1$\\
\\
\normalsize{$^1,2$Marine and Freshwater Research Centre, Galway-Mayo Institute of
	Technology (GMIT)}\\
\normalsize{Dublin Road, Galway, H91 T8NW, Ireland}\\
\normalsize{$^2$Centre for Environment, Fisheries and Aquaculture Science
	(Cefas),}\\
\normalsize{Pakefield Road, Lowestoft, Suffolk, NR33 0HT, UK}\\
\normalsize{$^3$North West Fisheries Science Center, NOAA,}\\
\normalsize{2725 Montlake Blvd E, Seattle,Washington, 98112, USA}\\
\\
\normalsize{$^\ast$To whom correspondence should be addressed; E-mail:
	paul.dolder@gmit.ie}
}

\date{}


%%%%%%%%%%%%%%%%%%%%%%%%%%%%%%%%%%%%%%%%

\begin{document}

% Double-space the manuscript.
\baselineskip24pt

\beginsupplement

% Make the title.
\maketitle

%%%%%%%%%%%%%%%%%%%%%%%%%%%%%%%%%%%%%%%%%
\begin{linenumbers}

\section*{Methods}

We use a geostatistical Vector Autoregressive Spatiotemporal model
	(VAST)\footnote{Software in the R statistical programming language can
	be found here: \url{www.github.com/james-thorson/VAST}} to implement a
delta-generalised linear mixed modelling (GLMM) framework that takes account of
spatio-temporal correlations among species through implementation of a spatial
dynamic factor analysis (SDFA). Spatial variation is captured through a
Gaussian Markov Random Field, while we model random variation among species and
years. Covariates affecting catchability (to account for differences between
fishing surveys) and density (to account for environmental preferences) can be
incorporated for predictions of presence and positive density. The following
briefly summarises the key methods implemented in the VAST framework. For full
details of the model the reader is invited directed to Thorson \textit{et al}
2017\cite{Thorson2017}.

\textbf{\textit{SDFA:}} A spatial dynamic factor analysis incorporates advances
in joint dynamic species models\cite{Thorson2017} to take account of
associations among species by modelling response variables as a multivariate
process. This is achieved through implementing a factor analysis decomposition
where common latent trends are estimated so that the number of common trends is
less than the number of species modelled. The factor coefficients are then
associated through a function for each factor that returns a positive or
negative association of one or more species with any location. Log-density of
any species is then be described as a linear combination of factors and
loadings: 

\begin{equation} \theta_{c}(s,t) = \sum_{j=1}^{n_{j}}
	L_{c,j}\psi_{j}(s,t) +\sum_{k=1}^{n_{k}} \gamma_{k,c}\chi_{k}(s,t)
\end{equation} 

Where $\theta_{c}(s,t)$ represents log-density for species $c$
at site $s$ at time $t$, $\psi_{j}$ is the coefficient for factor $j$,
$L_{c,j}$ the loading matrix representing association of species $c$ with
factor $j$ and $\gamma_{k,c}\chi_{k}(s,t)$ the linear effect of covariates at
each site and time\cite{Thorson2016b}. 

The factor analysis can identify community dynamics and where species have
similar spatio-temporal patterns, allowing inference of species distributions
and abundance of poorly sampled species through association with other species
and allows for computation of spatio-temporal correlations among
species\cite{Thorson2016b}.

\added[id = brutus]{We use the resultant factor analysis\deleted{is used} to identify
	community dynamics and drivers common among 18 species and results
	presented through transformation of the loading matrices using PCA
	rotation}. 

\textbf{\textit{Estimation of abundances:}} Spatio-temporal encounter
probability and positive catch rates are modelled separately with
spatio-temporal encounter probability modelled using a logit-link linear
predictor;

		\begin{equation}
			\begin{split}
			logit[p(s_{i},c_{i},t_{i})] =	\beta_{p}(c_{i},t_{i}) +
			& \sum\limits_{f=1}^{n_{\omega}} L_{\omega}(c_{i},f)
			\omega_{p}(s_{i},f) + \sum\limits_{f=1}^{n_{\varepsilon}}
			L_{\varepsilon}(c_{i},f) \varepsilon_{p}(s_{i},f,t_{i}) + \\ 
			& \sum\limits_{v=1}^{n_{v}}\delta_{p}(v)Q_{p}(c_{i}, v_{i})
		\end{split}
		\end{equation}

and positive catch rates modelling using a gamma- distribution\cite{Thorson2015a}.

		\begin{equation}
			\begin{split}
			log[r(s_{i},c_{i},t_{i})] = \beta_{r}(c_{i},t_{i}) +
			& \sum\limits_{f=1}^{n_{\omega}} L_{\omega}(c_{i},f)
			\omega_{r}(s_{i},f) +\sum\limits_{f=1}^{n_{\varepsilon}} 
			L_{\varepsilon}(c_{i},f) \varepsilon_{r}(s_{i},f,t_{i}) + \\
			& \sum\limits_{v=1}^{n_{v}}\delta_{r}(v) Q_{r}(c_{i}, v_{i})
			\end{split}
		\end{equation}

where $p(s_{i}, c_{i}, t_{i})$ is the predictor for encounter probability for
observation $i$, at location $s$ for species $c$ and time $t$ and $r(s_{i},
c_{i}, t_{i})$ is similarly the predictor for the positive density.
$\beta_{*}(c_{i},t_{i})$ is the intercept, $\omega_{*}(s_{i},c_{i})$ the
spatial variation at location $s$ for factor $f$, with $L_{\omega}(c_{i},f)$
the loading matrix for spatial covariation among species.
$\varepsilon_{*}(s_{i},c_{i},t_{i})$ is the linear predictor for
spatio-temporal variation, with $L_{\varepsilon}(c_{i}, f)$ the loading matrix
for spatio-temporal covariance among species and $\delta_{*}(c_{i}, v_{i})$ the
contribution of catchability covariates for the linear predictor with
$Q_{c_{i}, v_{i}}$ the catchability covariates for species $c$ and vessel $v$;
$*$ can be either $p$ for probability of encounter or $r$ for positive density.

The Delta-Gamma formulation is then:

\begin{equation}
	\begin{split}
	& Pr(C = 0) = 1 - p \\
	& Pr(C = c | c > 0) = p \cdot \frac{\lambda^{k}c^{k-1} \cdot exp(-\lambda c)}{\Gamma_{k}}
	\end{split}
\end{equation}

for the probability $p$ of a non-zero catch $C$ given a gamma distribution for
for the positive catch with a rate parameter $\lambda$ and shape parameter $k$.

\textbf{\textit{Spatio-temporal variation:}} The spatiotemporal variation is
modelled using Gaussian Markov Random Fields (GMRF) \replaced[id = Paul]{where
	observations are correlation in space}{where data is associated to
	nearby locations} through a Matérn covariance function with the
parameters estimated within the model. Here, the correlation decays smoothly
over space the further from the location and includes geometric anisotropy to
reflect the fact that correlations may decline in one direction faster than
another (e.g.  moving offshore)\cite{Thorson2013}.  The best fit estimated an
anisotropic covariance where the correlations were stronger in a north-east -
south-west direction, extending approximately 97 km and 140 km before
correlations for encounter probability and positive density reduced to
\textless 10 \%, respectively (Fig. S9). Incorporating the
spatiotemporal correlations among and within species provides more efficient
use of the data as inference can be made about poorly sampled locations from
the covariance structure.

A probability distribution for spatio-temporal variation in both encounter
probability and positive catch rate was specified, $\varepsilon_{*}(s,p,t)$,
with a three-dimensional multivariate normal distribution so that:

	\begin{equation}
		vec[\mathbf{E}_{*}(t)] \sim MVN(0,\mathbf{R}_{*} \otimes
		\mathbf{V}_{{\varepsilon}{*}})
	\end{equation}

Here, $vec[\mathbf{E}_{*}(t)]$ is the stacked columns of the matrices
describing $\varepsilon{*}(s,p,t)$ at every location, species and time,
$\mathbf{R}_{*}$ is a correlation matrix for encounter probability or positive
catch rates among locations and $\mathbf{V}_{*}$ a covariance matrix for
encounter probability or positive catch rate among species (modelled within the
factor analysis). $\otimes$ represents the Kronecker product so that the
correlation among any location and species can be computed\cite{Thorson2017}.
		
\textbf{\textit{Incorporating covariates}} Survey catchability (the relative
efficiency of a gear catching a species) was estimated as a fixed effect in the
model, $\delta_{s}(v)$, to account for differences in spatial fishing patterns
and gear characteristics which affect encounter and capture probability of the
sampling gear\cite{Thorson2014}. Parameter estimates (Fig. S10)
showed clear differential effects of surveys using otter trawl gears (more
effective for round fish species) and beam trawl gears (more effective for
flatfish species).

No fixed covariates for habitat quality or other predictors of encounter
probability or positive density were included. While incorporation may improve
the spatial predictive performance\cite{Thorson2017}, it was not found to be
the case here based on model selection with Akaike Information Criterion (AIC)
and Bayesian Information Criterion (BIC).

\textbf{\textit{Parameter estimation}} Parameter estimation was undertaken
through Laplace approximation of the marginal likelihood for fixed effects
while integrating the joint likelihood (which includes the probability of the
random effects) with respect to random effects. This was implemented using
Template Model Builder (TMB;\cite{Kristensen2015}) with computation through
support by the Irish Centre for High End Computing (ICHEC;
\url{https://www.ichec.ie}) facility.  

The model integrates data from seven fisheries independent surveys taking
account of correlations among species spatio-temporal distributions and
abundances to predict spatial density estimates consistent with the resolution
of the data. 

The model was been fit to nine species separated into adult and juvenile size
classes (Table S2) to seven survey series (Table S1)
in the Celtic Sea bound by 48$^{\circ}$ N to 52 $^{\circ}$ N latitude and 12
$^{\circ}$ W to 2$^{\circ}$ W longitude (Fig. S8) for the years
1990 - 2015 inclusive. 

The following steps were undertaken for data processing: i) data for survey
stations and catches were downloaded from ICES Datras
(\url{www.ices.dk/marine-data/data-portals/Pages/DATRAS.aspx}) or obtained
directly from the Cefas Fishing Survey System (FSS); ii) data were checked and
any tows with missing or erroneously recorded station information (e.g. tow
duration or distance infeasible) removed; iii) swept area for each of the
survey tows was estimated based on fitting a GAM to gear variables so that
Doorspread = s(Depth) + DoorWt + WarpLength + WarpDiameter + SweepLength and a
gear specific correction factor taken from the literature\cite{Piet2009}; iii)
fish lengths were converted to biomass (Kg) through estimating a von
bertalanffy length weight relationship, $Wt = a \cdot L^{b}$, fit to sampled
length and weight of fish obtained in the EVHOE survey and aggregated within
size classes (adult and juvenile). 

The final dataset comprised of estimates of catches (including zeros) for each
station and species and estimated swept area for the tow.

The spatial domain was setup to include 250 knots representing the Gaussian
Random Fields. The model was configured to estimate nine factors each to describe
the spatial and spatiotemporal encounter probability and positive density
parameters, with a logit-link for the linear predictor for encounter
probability and log-link for the linear predictor for positive density, with an
assumed gamma distribution.

Three candidate models were identified, i) a base model where the vessel
interaction was a random effect, ii) the base but where the vessel x species
effect was estimated as a fixed covariate, iii) with vessel x species effect
estimated, but with the addition of estimating fixed density covariates for
both predominant habitat type at a knot and depth. AIC and BIC model selection
favoured the second model (Table S3). The final model included
\replaced[id = Paul]{estimating 1,674 fixed parameters and predicting 129,276
	random effect values}{estimating 130,950 coefficients (1,674 fixed
	parameters and 129,276 random effect values)}.

Q-Q plots show good fit between the derived estimates and the data for positive
catch rates and between the predicted and observed encounter probability
(Figs. S11,S12).  Further, model outputs are consistent with
stock-level trends abundances over time from international assessments
(Fig. S13), yet also provide detailed insight into species
co-occurrence and the strength of associations in space and time. 


\bibliography{../JSDM}
\bibliographystyle{Science}

%% Figures %%


\begin{figure}[!ht]
	\label{fig:S1}
	\begin{subfigure}{0.5\textwidth}
	\includegraphics[width = \linewidth]{"../figures/Factor1_DepthO1"}
\end{subfigure}
\begin{subfigure}{0.5\textwidth}
	\includegraphics[width = \linewidth]{"../figures/Factor1_HabitatO1"}
\end{subfigure}
\caption{Left: Average spatial encounter probability factor 1 values correlated
	against; Left - Depth, Right - substrate type}

\end{figure}

%%

\begin{figure}[!ht]
	\label{fig:S2}
	
\begin{subfigure}{0.5\textwidth}
	\includegraphics[width = \linewidth]{"../figures/Factor1_DepthO2"}
\end{subfigure}
\begin{subfigure}{0.5\textwidth}
	\includegraphics[width = \linewidth]{"../figures/Factor1_HabitatO2"}
\end{subfigure}
\caption{Left: Average spatial positive density factor 1 values correlated
	against; Left- Depth, Right- substrate type}

\end{figure}

\begin{figure}[!ht]
	\label{fig:S3}
\begin{subfigure}{0.55\textwidth}
\includegraphics[width = \linewidth]{"../figures/Depth"}
\end{subfigure}	
\begin{subfigure}{0.45\textwidth}
\includegraphics[width = \linewidth]{"../figures/Substrate"}
\end{subfigure}
\caption{Left: Depth, Right: Substrate assigned to each spatial knot}

\end{figure}

%%%%%%%%

\begin{figure}
\begin{center}
	\includegraphics[width = 0.75\linewidth]{"../figures/Suppl - SpatioTempLoadingsEpsilon1"}
	\label{fig:S5}
	\caption{Spatial Loadings for first three factors every five years for
	spatio-temporal encounter probability}
	\end{center}
\end{figure}

\begin{figure}[!ht]
\begin{center}
	\includegraphics[width = 0.75\linewidth]{"../figures/Suppl - SpatioTempLoadingsEpsilon2"}
	\label{fig:S7}
	\caption{Spatial Loadings for first three factors every five years for
	spatio-temporal density}
	\end{center}
\end{figure}


\begin{figure}[!ht]
\begin{center}
	\includegraphics[width = 0.55\linewidth]{"../figures/Suppl - TempAndFactors"}
	\label{fig:S6}
	\caption{Association of temperature and knots (individual lines; top)
		with Spatio-temporal factor loadings for encounter probability
	(middle) and density (bottom)}
	\end{center}
\end{figure}



\begin{figure}[!ht]
\begin{center}
	\includegraphics[width = \linewidth]{"../figures/Suppl - Epsilon1Epsilon2_Correlations_blank"}
	\label{fig:S8}
	\caption{Inter-species correlations for (a) spatio-temporal encounter
		probability and b) spatio-temporal density.  Species-groups are
		clustered into three groups based on a hierarchical clustering
		method with non-significant correlations (those where the
		Confidence Interval spanned zero) left blank}
	\end{center}
\end{figure}

\begin{figure}[!ht]
\begin{center}
	\includegraphics[width = 0.9\linewidth]{"../figures/AreaMap"}
	\label{fig:S9}
	\caption{Spatial bounds of case study area}
	\end{center}
\end{figure}

\begin{figure}[!ht]
\begin{center}
	\includegraphics[width = 0.9\linewidth]{"../figures/Aniso"}
	\label{fig:S10}
	\caption{Estimates of distances at 10 \% correlation from the Matérn
		covariance function for encounter probability and positive
		catch rates}
	\end{center}
\end{figure}


\begin{figure}[!ht]
\begin{center}
	\includegraphics[width = 0.85\linewidth]{"../figures/Suppl - QEstimatesALL"}
	\label{fig:S11}
	\caption{Fixed effect estimates for surveys for each species-group. Note
	all values within a species-group are relative to the CEXP survey}
	\end{center}
\end{figure}

\begin{figure}[!ht]
\begin{center}
	\includegraphics[width = 0.9\linewidth]{"../figures/Diag--Encounter_prob"}
	\label{fig:S12}
	\caption{Model diagnostics output showing correlation between the
		predicted encounter probability and the data}
	\end{center}
\end{figure}

\begin{figure}[!ht]
\begin{center}
	\includegraphics[width = 0.9\linewidth]{"../figures/Q-Q_plot"}
	\label{fig:S13}
	\caption{Model diagnostics output showing the Q-Q plot for the positive
	catch rates}
	\end{center}
\end{figure}

\begin{figure}[!ht]
\begin{center}
	\includegraphics[width =0.9\linewidth]{"../figures/RealativeIndexVRelativeAssessSSB"}
	\label{fig:S14}
	\caption{Comparison between the standardised index from the VAST
		output and the standardised spawning stock biomass (SSB) from
		the assessments for cod, haddock and whiting}
	\end{center}
\end{figure}

%% Tables %%
\begin{table}[!ht]
	\caption{List of survey codes, names and brief description}
	\center
	\begin{tabular}{ p{3cm} p{4cm} p{4cm} p{3cm} }
		\hline
		Survey code    & Name 	& Gear & Temporal extent \\
		\hline
		CEXP           & Celtic Explorer (IE)   & Otter trawl & 2003 - 2015 \\
		CARLHELMAR     & Carlhelmar (UK)	& Commercial beam trawl & 1989 - 2013 \\
		NWGFS          & North West groundfish survey (UK) & Beam trawl & 1988 - 2015 \\
		Q1SWBEAM       & Quarter 1 south-west beam trawl survey (UK) 	& beam trawl & 2006 - 2015 \\
		Q4SWIBTS       & Quarter 4 south-west international bottom trawl survey (UK) & Otter trawl & 2003 - 2010 \\
		THA2           & EVHOE survey on Thalasa (FR) & Otter trawl & 1997 - 2015 \\
		WCGFS          & Western channel groundfish survey (UK) & Otter
		trawl (Portugese high headline) & 1982 - 2004 \\
		\hline
	\end{tabular}
\end{table}


\begin{table}[!ht]
	\caption{List of species codes, names and minimum conservation
		reference size used to separate juvenile and adult fish}
	\center
	\begin{tabular}{ p{3cm} p{4cm} p{6cm} p{2cm} }
		\hline
		Species code & Common name              & Species & MCRS (cm) \\
		\hline
		juv          & Juvenile                 & \\
		adu          & Adult                    & \\
		\hline
		bud          & Black bellied anglerfish & \textit{Lophius
			budgessa} &  32* \\
		cod          & Atlantic cod             & \textit{Gadus morhua}
		& 35 \\
		had          & Atlatic haddock          & \textit{Melanogrammus
			aeglefinus} & 30 \\
		hke          & Atlantic hake            & \textit{Merluccius
			merluccius} & 27 \\
		meg          & Megrim                   & \textit{Lepidorhombus
			whiffiagonis} & 20 \\
		pisc         & White bellied anglerfish & \textit{Lophius
			piscatorius}	& 32* \\
		ple          & European Plaice          & \textit{Pleuronectes
			platessa} & 27 \\
		sol          & Common sole              & \textit{Solea solea}
		& 24 \\
		whg          & Atlantic whiting         & \textit{Merlangius
			merlangus} & 27 \\
		\hline
	\end{tabular}
	*Anglerfish species estimated based on a 500g minimum marketing weight

\end{table}


\begin{table}[!ht]
	\caption{Description of model variants and AIC / BIC }
	\begin{tabular} { {p}{1cm} p{4cm} p{2cm} p{2cm} p{2cm} p{2cm} }
		\hline
		Model & Description & No fixed parameters & No random
		parameters & AIC & BIC \\
		\hline
		H0 & Vessel random effects, no covariates & 1462 & 129276 &
		125954 & 140187 \\
		H1 & With fixed gear effect, no density covariates & 1674 &
		129276 & 116012 & 132309 \\
		
		H2 & With fixed gear effect, substrate and depth density
		covariates & 1688 & 129276 & 116013 & 132446 \\
		\hline
	\end{tabular}
\end{table}

\end{linenumbers}

\end{document}
