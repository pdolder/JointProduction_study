%% Set up 
\documentclass{article}
\usepackage[margin=1in]{geometry} % Required to make the margins smaller to fit
% more content on each page 
\usepackage{lineno} % add some line nos to aid reading
\usepackage[utf8]{inputenc}
\usepackage{enumitem}
\usepackage[hyphens]{url} % for breaking url's in the bib
\usepackage{amsmath}
\usepackage{graphicx}

% change references to parenthesis (from square brackets)
% figures here	
\graphicspath{{figures/}}
%%%%%%%%%%%%%%%%%%%%%%%%%%%%%%%%%%%%%%%%

\begin{document}

\begin{figure}
\begin{center}
	\includegraphics[width = \linewidth]{Omega1Omega2_Correlations}
	\label{fig:1}
	\caption{Inter-species correlations for (a) spatial encounter
		probability over all years and (b) spatial density.
		Species-groups are clustered into three groups based on a
		hierarchical clustering method with NOTE: Final figure will
		have non-sig correlations blanked}
\end{center}
\end{figure}

\begin{figure}
\begin{center}
	\includegraphics[width=\linewidth]{PCAstyle_Plots_Spatial_FishPics_Leg}
	\label{fig:2}
	\caption{Position of each species-group on the first two loading axes from the
	factor analysis for (a) spatio-temporal encounter probability and (b)
	spatio-temporal density}
\end{center}
\end{figure}

\begin{figure}
\begin{center}
	\includegraphics[width=\linewidth]{SpatialFactorLoadingsOmega1}
	\label{fig:3}
	\caption{Spatial Factor loadings for the average spatial encounter
		probability on the first 3 factors}
\end{center}
\end{figure}

\begin{figure}
\begin{center}
	\includegraphics[width = \linewidth]{DensityFigureswithCCRevised2}
	\label{fig:4}
	\caption{Differences in spatial density for pairs of species and
		expected catch rates for two different gears at three different
	locations}
\end{center}
\end{figure}

\begin{figure}
\begin{center}
	\includegraphics[width=\linewidth]{TechnicalEfficiency}
	\label{fig:5}
	\caption{Example of technical efficiency space}
\end{center}
\end{figure}

\end{document}
