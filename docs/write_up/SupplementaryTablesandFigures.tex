%%%%%%%%%%%%%%%%%%%%%%%%%%%%%%%%%%%%%%%%
%% Set up 
\documentclass{article}
\usepackage[margin=1in]{geometry} % Required to make the margins smaller to fit
% more content on each page 
\usepackage{lineno} % add some line nos to aid reading
\usepackage[utf8]{inputenc}
\usepackage{enumitem}
\usepackage[hyphens]{url} % for breaking url's in the bib
\usepackage{amsmath}
\usepackage{graphicx}
\usepackage{subcaption}

% cite is loaded from nature class
\setlist{itemsep=1pt} % This controls spacing between items in the lists
\setlength\parindent{0pt} % Removes all indentation from paragraphs

% figures here	
\graphicspath{figures/}

%% To allow supplementary table and figure numbering
%% http://bytesizebio.net/2013/03/11/adding-supplementary-tables-and-figures-in-latex/

\newcommand{\beginsupplement}{%
		        \setcounter{table}{0}
		        \renewcommand{\thetable}{S\arabic{table}}%
		        \setcounter{figure}{0}
		        \renewcommand{\thefigure}{S\arabic{figure}}%
			     }



%%%%%%%%%%%%%%%%%%%%%%%%%%%%%%%%%%%%%%%%

\begin{document}

\beginsupplement


%% Figures %%

\begin{figure}[!ht]
	\label{fig:S1}
	\begin{subfigure}{0.5\textwidth}
	\includegraphics[width = \linewidth]{"figures/Factor1_DepthO1"}
\end{subfigure}
\begin{subfigure}{0.5\textwidth}
	\includegraphics[width = \linewidth]{"figures/Factor1_HabitatO1"}
\end{subfigure}
\caption{Left: Spatial Omega 1 correlated against Depth, Right: Omega 1
	correlated against substrate type}

\end{figure}

%%

\begin{figure}[!ht]
	\label{fig:S2}
	
\begin{subfigure}{0.5\textwidth}
	\includegraphics[width = \linewidth]{"figures/Factor1_DepthO2"}
\end{subfigure}
\begin{subfigure}{0.5\textwidth}
	\includegraphics[width = \linewidth]{"figures/Factor1_HabitatO2"}
\end{subfigure}
\caption{Left: Spatial Omega 2 correlated against Depth, Right: Omega 2
	correlated against substrate type}

\end{figure}

\begin{figure}[!ht]
	\label{fig:S3}
\begin{subfigure}{0.55\textwidth}
\includegraphics[width = \linewidth]{"figures/Depth"}
\end{subfigure}	
\begin{subfigure}{0.45\textwidth}
\includegraphics[width = \linewidth]{"figures/Substrate"}
\end{subfigure}
\caption{Left: Depth, Right: Substrate assigned to each spatial knot}

\end{figure}

%%%%%%%%

\begin{figure}
\begin{center}
	\includegraphics[width = 0.8\linewidth]{"figures/Suppl - SpatioTempLoadingsEpsilon1"}
	\label{fig:S5}
	\caption{Spatial Loadings for first three factors every five years for
	spatio-temporal encounter probability}
	\end{center}
\end{figure}

\begin{figure}[!ht]
\begin{center}
	\includegraphics[width = 0.8\linewidth]{"figures/Suppl - SpatioTempLoadingsEpsilon2"}
	\label{fig:S7}
	\caption{Spatial Loadings for first three factors every five years for
	spatio-temporal density}
	\end{center}
\end{figure}


\begin{figure}[!ht]
\begin{center}
	\includegraphics[width = 0.6\linewidth]{"figures/Suppl - TempAndFactors"}
	\label{fig:S6}
	\caption{Association of temperature and knots (individual lines; top)
		with Spatio-temporal factor loadings for encounter probability
	(middle) and density (bottom)}
	\end{center}
\end{figure}



\begin{figure}[!ht]
\begin{center}
	\includegraphics[width = \linewidth]{"figures/Suppl - Epsilon1Epsilon2_Correlations_blank"}
	\label{fig:S8}
	\caption{Inter-species correlations for (a) spatio-temporal encounter
		probability and b) spatio-temporal density.  Species-groups are
		clustered into three groups based on a hierarchical clustering
		method with non-significant correlations (those where the
		Confidence Interval spanned zero) left blank}
	\end{center}
\end{figure}

\begin{figure}[!ht]
\begin{center}
	\includegraphics[width = 0.9\linewidth]{"figures/AreaMap"}
	\label{fig:S9}
	\caption{Spatial bounds of case study area}
	\end{center}
\end{figure}

\begin{figure}[!ht]
\begin{center}
	\includegraphics[width = 0.9\linewidth]{"figures/Aniso"}
	\label{fig:S10}
	\caption{Estimates of distances at 10 \% correlation from the Matérn
		covariance function for encounter probability and positive
		catch rates}
	\end{center}
\end{figure}


\begin{figure}[!ht]
\begin{center}
	\includegraphics[width = 0.9\linewidth]{"figures/Suppl - QEstimatesALL"}
	\label{fig:S11}
	\caption{Fixed effect estimates for surveys for each species-group. Note
	all values within a species-group are relative to the CEXP survey}
	\end{center}
\end{figure}

\begin{figure}[!ht]
\begin{center}
	\includegraphics[width = 0.9\linewidth]{"figures/Diag--Encounter_prob"}
	\label{fig:S12}
	\caption{Model diagnostics output showing correlation between the
		predicted encounter probability and the data}
	\end{center}
\end{figure}

\begin{figure}[!ht]
\begin{center}
	\includegraphics[width = 0.9\linewidth]{"figures/Q-Q_plot"}
	\label{fig:S13}
	\caption{Model diagnostics output showing the Q-Q plot for the positive
	catch rates}
	\end{center}
\end{figure}

\begin{figure}[!ht]
\begin{center}
	\includegraphics[width =0.9\linewidth]{"figures/RealativeIndexVRelativeAssessSSB"}
	\label{fig:S14}
	\caption{Comparison between the standardised index from the VAST
		output and the standardised spawning stock biomass (SSB) from
		the assessments for cod, haddock and whiting}
	\end{center}
\end{figure}

%% Tables %%
\begin{table}[!ht]
	\caption{List of survey codes, names and brief description}
	\center
	\begin{tabular}{ p{3cm} p{4cm} p{4cm} p{3cm} }
		\hline
		Survey code    & Name 	& Gear & Temporal extent \\
		\hline
		CEXP           & Celtic Explorer (IE)   & Otter trawl & 2003 - 2015 \\
		CARLHELMAR     & Carlhelmar (UK)	& Commercial beam trawl & 1989 - 2013 \\
		NWGFS          & North West groundfish survey (UK) & Beam trawl & 1988 - 2015 \\
		Q1SWBEAM       & Quarter 1 south-west beam trawl survey (UK) 	& beam trawl & 2006 - 2015 \\
		Q4SWIBTS       & Quarter 4 south-west international bottom trawl survey (UK) & Otter trawl & 2003 - 2010 \\
		THA2           & EVHOE survey on Thalasa (FR) & Otter trawl & 1997 - 2015 \\
		WCGFS          & Western channel groundfish survey (UK) & Otter
		trawl (Portugese high headline) & 1982 - 2004 \\
		\hline
	\end{tabular}
\end{table}


\begin{table}[!ht]
	\caption{List of species codes, names and minimum conservation
		reference size used to separate juvenile and adult fish}
	\center
	\begin{tabular}{ p{3cm} p{4cm} p{6cm} p{2cm} }
		\hline
		Species code & Common name              & Species & MCRS (cm) \\
		\hline
		juv          & Juvenile                 & \\
		adu          & Adult                    & \\
		\hline
		bud          & Black bellied anglerfish & \textit{Lophius
			budgessa} &  32* \\
		cod          & Atlantic cod             & \textit{Gadus morhua}
		& 35 \\
		had          & Atlatic haddock          & \textit{Melanogrammus
			aeglefinus} & 30 \\
		hke          & Atlantic hake            & \textit{Merluccius
			merluccius} & 27 \\
		meg          & Megrim                   & \textit{Lepidorhombus
			whiffiagonis} & 20 \\
		pisc         & White bellied anglerfish & \textit{Lophius
			piscatorius}	& 32* \\
		ple          & European Plaice          & \textit{Pleuronectes
			platessa} & 27 \\
		sol          & Common sole              & \textit{Solea solea}
		& 24 \\
		whg          & Atlantic whiting         & \textit{Merlangius
			merlangus} & 27 \\
		\hline
	\end{tabular}
	*Anglerfish species estimated based on a 500g minimum marketing weight

\end{table}


\begin{table}[!ht]
	\caption{Description of model variants and AIC / BIC }
	\begin{tabular} { {p}{1cm} p{4cm} p{2cm} p{2cm} p{2cm} p{2cm} }
		\hline
		Model & Description & No fixed parameters & No random
		parameters & AIC & BIC \\
		\hline
		H0 & Vessel random effects, no covariates & 1462 & 129276 &
		125954 & 140187 \\
		H1 & With fixed gear effect, no density covariates & 1674 &
		129276 & 116012 & 132309 \\
		
		H2 & With fixed gear effect, substrate and depth density
		covariates & 1688 & 129276 & 116013 & 132446 \\
		\hline
	\end{tabular}
\end{table}


\end{document}
