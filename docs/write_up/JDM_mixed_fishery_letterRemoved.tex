%%
%%%%%%%%%%%%%%%%%%%%%%%%%%%%%%%%%%%%%%%%
%% Set up 
\documentclass{nature}
\usepackage[margin=1in]{geometry} % Required to make the margins smaller to fit
% more content on each page 
\usepackage{lineno} % add some line nos to aid reading
\usepackage[utf8]{inputenc}
\usepackage{enumitem}
\usepackage[hyphens]{url} % for breaking url's in the bib
\usepackage{amsmath}
\usepackage{graphicx}
\usepackage[]{changes}
\definechangesauthor[color = red]{coilin}
\definechangesauthor[color = blue]{Paul}
\definechangesauthor[color = purple]{brutus}

% cite is loaded from nature class
\setlist{itemsep=1pt} % This controls spacing betweencitems in the lists
\setlength\parindent{0pt} % Removes all indentation from paragraphs
% figures here	
\graphicspath{{figures/}}
\renewcommand{\familydefault}{\sfdefault}
%%%%%%%%%%%%%%%%%%%%%%%%%%%%%%%%%%%%%%%%

\title{Spatial separation of catches in highly mixed fisheries}

\author{Paul J. Dolder$^{1,2}$ \& James T. Thorson$^3$ \& Cóilín Minto$^1$}

%%%%%%%%%%%%%%%%%%%%%%%%%%%%%%%%%%%%%%%%%
\begin{document}
\maketitle

\begin{affiliations}
\item Marine and Freshwater Research Centre, Galway-Mayo Institute of
	Technology (GMIT), Dublin Road, Galway, H91 T8NW, Ireland 
\item Centre for Environment, Fisheries and Aquaculture Science (Cefas),
	Pakefield Road, Lowestoft, Suffolk, NR33 0HT, UK
\item North West Fisheries Science Center, NOAA, 2725 Montlake Blvd E, Seattle,
	Washington, 98112, USA
\end{affiliations}

%%%%%%%%%%%%%%%%%%%%%%%%%%%%%%%%%%%%%%%%%
\begin{linenumbers}

\begin{abstract} 
%% based on: https://www.nature.com/nature/authors/gta/2c_Summary_para.pdf
%% Basic introduction 
Mixed fisheries capture a mix of species at the same time and are the dominant
type of fishery worldwide.  Overexploitation in mixed fisheries occurs when
catches continue for available quota species while low quota species are
discarded\cite{Batsleer2015}. As EU fisheries management moves to count all fish
caught against quota (the `landings obligation'), the challenge is to catch
available quota within new constraints, else lose productivity.  
%% More detailed background
A mechanism for decoupling exploitation of species caught together is spatial
targeting, but this remains challenging due to complex fishery and population
dynamics in space and time\cite{Branch2008, Dunn2014a}. 
%% General problem
How far spatial targeting can go to practically separate species is often
unknown and anecdotal.
%% Summarising the main result
Here we develop a dimension-reduction framework based on joint species
distribution modelling (spatial dynamic factor analysis) to understand how
spatial community and fishery dynamics interact to determine species and size
composition.  
%%	 Two or three sentences explaining what the main result
In the example application to the highly mixed fisheries of the Celtic Sea,
clear common spatial patterns emerge for three distinct species-groups and,
while distribution varies inter-annually, the same species-groups are
consistently found in higher densities together, with more subtle differences
within species-groups - where spatial separation may not be practically
possible. 
%%  general context
We highlight the importance of dimension reduction techniques to focus
management discussion on axes of maximal separation in space and time. We
propose that spatiotemporal modelling of available data is a scientific
necessity to address the pervasive and nuanced challenges of managing mixed
fisheries.  
\end{abstract}

%%%%%%%%%%%%%%%%%%%%%%%%%%%%%%%%%%%%%%%%%
\section*{}
\subsection{Mixed fisheries and the EU landings obligation} 

 Efforts to reduce exploitation rates in commercial fisheries
have begun the process of rebuilding depleted fish populations\cite{Worm2009}.
Improved fisheries management  can increase
population sizes and allow increased sustainable catches, yet fisheries catch
globally remains stagnant\cite{FAO2016}. With future increased demand for fish
protein  there is an important role for well managed
fisheries in supporting future food security \cite{Mcclanahan2015}
  necessitating
	 fisheries are managed efficiently to
maximise productivity.

A  challenge in realising increased catches
from rebuilt populations is maximising yields from mixed fisheries
\cite{Branch2008, Kuriyama2016, Ulrich2016}. In mixed fisheries managed by individual quotas,
if catches do not match available stock quotas,
either a vessel must stop fishing when the first quota is reached (the `choke'
species) or overexploitation of the weaker species occurs while fishers
 catch more healthy species and throw back (`discard') the
fish for which they have no quota\cite{Batsleer2015}.  There is a pressing need for scientific tools which simplify the
complexities of mixed fisheries to help avoid
	discarding. 

Sustainability of European fisheries has been hampered by this `mixed fishery
problem' for decades with large-scale discarding\cite{Uhlmann2014}.   Under the EU Common Fisheries
Policy (CFP) reform of 2012 , by 2019 all fish that are caught are due to be
counted against the respective stock quota.
    Unless
fishers can avoid catch of unwanted species they will have to stop fishing when
reaching their first restrictive quota  introducing a
 significant cost to  under-utilised
quota\cite{Ulrich2016} and  a strong incentive to mitigate
such losses\cite{Condie2013}. The ability  to align
 catch with available quota depends on being able to exploit
target species while avoiding unwanted catch, either by switching fishing
method , changing technical gear
characteristics , or the
timing and location of fishing activity\cite{vanPutten2012a}. 

Spatiotemporal  measures  have been applied to reduce unwanted catch with
varying degrees of success\cite{Needle2011, Dunn2014a}, partly because
 they have  been
targeted at individual species without considering associations  among several species. Highly mixed fisheries are complex
with spatial, technological and community interactions ;
  Our goal is to develop a
framework for understanding these complexities. We do so by implementing a
spatio-temporal dimension reduction method and use  results to
draw inference on the fishery-community dynamics, creating a framework to
identify trends common among species  and describe
the potential for and limitations of spatial
measures to mitigate
unwanted catches in highly mixed fisheries.


\subsection{Framework for analysing spatio-temporal mixed fisheries
	interactions}

 We 
characterise the spatiotemporal dynamics of key components of a fish community
by  implementing a factor analysis decomposition
to describe trends in spatiotemporal dynamics of the different species as a
function of latent variables \cite{Thorson2015} representing spatial variation
(9 factors;  'average' spatial variation) and
spatio-temporal variation (9 factors) for encounter probability and positive
catch rates ( 'positive density')
separately\cite{Thorson2015a}. This allows us to  take account of how the factors contribute to affect
catches of the species in mixed fisheries.  Gaussian Markov
Random Fields (GMRFs)  capture spatial and temporal dependence
within and among species groups for both encounter probability and positive
density\cite{Thorson2013}.  Fixed effects  account for
systematic differences driving encounter and catches such as
differences in sampling efficiency (a.k.a.  catchability), while random effects
capture the spatio-temporal dynamics of the fish community.

\subsection{Dynamics of Celtic Sea fisheries}

 The Celtic Sea is a temperate sea where fisheries are spatially
and temporally complex\cite{Ellis2000, Gerritsen2012}.
Close to 150 species have been identified in the commercial catches of the
Celtic Sea, with approximately 30 species dominating the catch\cite{Mateo2016}.
We parametrise our  model using catch data from seven
fisheries-independent surveys undertaken  over the
period 1990 - 2015 (Table S1) and include nine of the main commercial species
(see Table S2, Figure 2)  which make up \textgreater 60 \%
of landings by towed fishing gears for the area (average 2011 -
2015;\cite{STECF2017}). Each species was separated into juvenile and adult size
classes based on their legal minimum conservation reference size (Table S2).



\subsection{Common average spatial patterns driving species associations} A
spatial dynamic factor analysis decomposes the dominant spatial patterns
driving differences in encounter probability and positive density. The first
three factors  account for 83.7\% of the between
species variance in average encounter probability and 69\% of the between
species variance in average positive density. A clear spatial pattern can been
seen both for encounter probability and positive density, with a positive value
associated with the first factor in the inshore north easterly part of the
Celtic Sea into the Bristol Channel and Western English Channel, moving to a
negative value offshore in the south-westerly waters (Figure 1).  The species
loadings coefficients show plaice, sole and whiting to be positively associated
with the first factor for encounter probability while the other species are
negatively associated. For average positive density, positive associations are
also found for haddock and juvenile cod.  

On the second spatial factor for encounter probability a north / south split
can be seen at approximately 49$^{\circ}$ N while positive density is more
driven by a positive value in the deeper westerly waters as well as some
inshore areas. Species values for the second factor indicate there are positive
associations for juvenile monkfish (\emph{L. piscatorius}), juvenile hake,
juvenile megrim, plaice and juvenile whiting with average positive density,
which may reflect two different spatial distributions in the more offshore and
in the inshore areas (Figure 1).

On the third factor, there is a positive association with the easterly waters
for encounter probability and negative with the westerly waters. This
 splits the
roundfish species cod, haddock and whiting which  have a positive
association with the third factor for average encounter probability from the
rest of the species .  Positive
density is driven by a north / south split (Figure 1), with positive values in
the northerly areas. Juvenile monkfish (\emph{L.  budgessa} and \emph{L.
	piscatorius}), cod, juvenile haddock, hake, adult plaice and whiting
are also positively associated with the third factor in the north while adult
monkfish (\emph{L. budgessa} and \emph{L.  piscatorius}), adult haddock,
megrims, juvenile plaice and sole are negatively associated  (Figure 1).

   





\subsection{Time varying species distributions, but
		stability within species groups}   Inter-annual differences in factor
coefficients show less structure (Figures S5, S6). These inter-annual
differences are important as they reflect the ability of fishers to predict
where they can target species from one year to the next.  Common
patterns in spatiotemporal factor coefficients among species   drive spatiotemporal distributions of megrim, anglerfish
species and hake (the deeper water species, species-group negatively associated
with the second axes of Figure 2a) and the roundfish and flatfish
(species-group more positively associated with the second axes of Figure 2a).
For spatio-temporal positive density (Figure 2b) cod, haddock and whiting (the
roundfish species) are separated from plaice, sole (the flatfish) and deeper
water species.   

\subsection{Three clusters of species show similar spatial patterns}
To gain greater insight into the community
dynamics we considered how species covary in space and time through among
species correlations. Pearson correlation coefficients for the modelled average
spatial encounter probability (Figure 3a) show clear strong associations
between adult and juvenile size classes for all species (\textgreater 0.75 for
all species except hake, 0.56).   Hierarchical
clustering identified the same three common groups  with roundfish (cod,
haddock, whiting) , flatfish (plaice and sole)  and species found in the deeper
waters (hake, megrim and both anglerfish species)  showing strong
intra-group correlations indicating similar spatial distributions.

 Correlation coefficients for the average
positive density also have strong associations among  roundfish
(Figure 3b) 




\subsection{Subtle differences in distributions may be important to separate
	catches within groups }: If production from
mixed fisheries is to be maximised, decoupling catches of species between and
within the groups will be key. For example, asking where the maximal separation
in the densities of two coupled species is likely to occur?  We map the difference in spatial distribution
 for each pair of species within a species-group for a
single year (2015; Figure 4)  to facilitate
discussion on maximal separation, for example, between difficult to separate
species such as haddock and whiting (Figure 4c). 