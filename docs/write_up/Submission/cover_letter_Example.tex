\documentclass[a4paper,12pt]{letter}
\usepackage{comment}
% Some of the article customisations are relevant for this class

%\name{Coil\'in Minto} % To be used for the return address on the envelope
%\signature{} % Goes after the closing (ie at the end of the letter, with space for a signature)
\address{\vspace*{-1in} \\C\'oil\'in Minto,\\Department of Biology,\\ Dalhousie University,\\ Halifax, Nova Scotia,\\ Canada B3H 4J1.\\Email: mintoc@mathstat.dal.ca\\ Phone: (902) 494 2146\\ Fax: (902) 494 3736}
% Alternatively, these may be set on an individual basis within each letter environment.

\makelabels % this command prints envelope labels on the final page of the document
\begin{document}

\begin{letter}{Dr. Rory Howlett,\\Deputy Biological Sciences Editor,\\
Nature Editorial Offices,\\The Macmillan Building,\\ 4 Crinan Street, \\ London N1 9XW,  \\ England.}

\opening{Dear Dr. Howlett,} % Rory Howlett eg Hello.

Enclosed is our manuscript, "Survival variability and population density". Please accept it as a candidate for publication as a Letter to \textit{Nature}.

{\bf Summary of appeal to a general scientific audience}\\
Understanding how population stochasticity arises is a fundamental ecological question with relevance to managing living resources, promoting recovery from depletion, and avoiding extinction. Recent results published in \textit{Nature} (\textit{Fishing elevates the variability in the abundance of exploited species}, vol. 443, 2006) showed that population abundance is made more variable by fishing. The mechanism by which this occurs is unknown. Here we show for the first time, by developing new ecological theory and then testing it empirically in a global meta-analysis of fisheries data, that density-dependent survival variability increases at low population sizes. In so doing, we not only address mechanisms to explain contemporary issues of human-induced population changes but also provide an exciting and novel addition to the long-standing question in population ecology of how populations are regulated. Recent high-profile papers (e.g. \textit{On the Regulation of Populations of Mammals, Birds, Fish, and Insects}, \textit{Science} vol.309, 2005) question how populations are regulated but these contemporary approaches are often hindered by markedly variable data. A previously overlooked aspect of the regulation debate is the
possibility that the pattern of variability, specifically, its strength as a
function of population size, is more than ``noise", and reveals much about the characteristics of regulation. The results are compelling. For example, we demonstrate a hitherto undiscovered behaviour that all commonly applied population growth models that incorporate density-dependence, carry not only changes in the mean abundance but, moreover, undergo changes in the population variability. 
Disregard for this large heteroscedastic component represents a fundamental weakness in population ecology. We believe 
that our approach and results will be of interest not only to ecologists but also to a multidisciplinary readership in other fields confronted by markedly variable data.\\
This paper is our original unpublished work and it has not been submitted to any other journal for reviews.\\
Suggested referees:
\begin{itemize}
\item{Prof. Russell Lande, Department of Biology 0116, University of California, San Diego, La Jolla, California 92093, USA.}
\item{Prof. Peter Turchin, Department of Ecology and Evolutionary Biology, University of Connecticut,
75 N. Eagleville Road, U-43, Storrs, CT 06269-3043, USA.}
\item{Prof. Richard Sibly, School of Animal and Microbial Sciences, University of Reading, Whiteknights, Reading RG6 6AJ, UK.}
\end{itemize}
{\bf Summary of appeal to  a non-scientific audience}\\
Population sizes of insects, fish, birds, and mammals vary greatly from year to year. Consider the varying numbers of a favourite songbird returning to a feeding table over time. It is a major goal of ecology to understand why this occurs. Traditionally, highly variable data have proved to be an obstacle in unravelling abundance patterns. This has lead to divergent theories as to whether population growth will decrease at large numbers because of limited resources. Here we show that if populations are regulated according to their size, remarkable patterns should be present in the variable data; the same data which were previously thought of as an obstacle. Testing these predictions on a global database of fish species, we have drawn exciting conclusions which  lend considerable support to populations being regulated through their size. Currently, we witness many species at historically low numbers so apart from ecological theory why should we care about regulation at large numbers? Our predictions relate to all population sizes and reveal, in particular, that variance will continue to increase, the lower the population size becomes. Focusing solely on the average numbers of a population will miss this fact and increase the risk of numbers plummeting to extinction.

{\bf Manuscript details}\\
We estimate a final draft to be 3,200 words or 3.5 pages in \textit{Nature}. There are three figures comprising 19 panels. The desired figure sizes are height x width in millimeters:\\
Figure 1: 88.9 x 86.36\\
Figure 2: 106.17 x 170.2\\
Figure 3: 106.17 x 170.2\\
A Supplementary Information section is included.

\closing{Yours sincerely,\\ \vspace{.07in} C\'oil\'in Minto and Wade Blanchard} % eg Regards,
\vspace{-.5in}
Please note we are very sorry to you inform that our co-author Prof. Ransom Myers passed away on March 27, 2007. He was fully aware of all aspects of the manuscript prior to his illness, beginning with hospitalization on November 26, 2006.\\

\end{letter}
\end{document}
