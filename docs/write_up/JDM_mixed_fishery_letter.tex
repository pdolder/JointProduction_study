%%
%%%%%%%%%%%%%%%%%%%%%%%%%%%%%%%%%%%%%%%%
%% Set up 
\documentclass{nature}
\usepackage[margin=1in]{geometry} % Required to make the margins smaller to fit
% more content on each page 
\usepackage{lineno} % add some line nos to aid reading
\usepackage[utf8]{inputenc}
\usepackage{enumitem}
\usepackage[hyphens]{url} % for breaking url's in the bib
\usepackage{amsmath}
\usepackage{graphicx}
\usepackage[final]{changes}
\definechangesauthor[color = red]{coilin}
\definechangesauthor[color = blue]{Paul}
\definechangesauthor[color = purple]{brutus}

% cite is loaded from nature class
\setlist{itemsep=1pt} % This controls spacing betweencitems in the lists
\setlength\parindent{0pt} % Removes all indentation from paragraphs
% figures here	
\graphicspath{{figures/}}
\renewcommand{\familydefault}{\sfdefault}
%%%%%%%%%%%%%%%%%%%%%%%%%%%%%%%%%%%%%%%%

\title{Spatial separation of catches in highly mixed fisheries}

\author{Paul J. Dolder$^{1,2}$ \& James T. Thorson$^3$ \& Cóilín Minto$^1$}

%%%%%%%%%%%%%%%%%%%%%%%%%%%%%%%%%%%%%%%%%
\begin{document}
\maketitle

\begin{affiliations}
\item Marine and Freshwater Research Centre, Galway-Mayo Institute of
	Technology (GMIT), Dublin Road, Galway, H91 T8NW, Ireland 
\item Centre for Environment, Fisheries and Aquaculture Science (Cefas),
	Pakefield Road, Lowestoft, Suffolk, NR33 0HT, UK
\item North West Fisheries Science Center, NOAA, 2725 Montlake Blvd E, Seattle,
	Washington, 98112, USA
\end{affiliations}

%%%%%%%%%%%%%%%%%%%%%%%%%%%%%%%%%%%%%%%%%
\begin{linenumbers}

\begin{abstract} 
%% based on: https://www.nature.com/nature/authors/gta/2c_Summary_para.pdf
%% Basic introduction 
Mixed fisheries capture a mix of species at the same time and are the dominant
type of fishery worldwide.  Overexploitation in mixed fisheries occurs when
catches continue for available quota species while low quota species are
discarded\cite{Batsleer2015}. As EU fisheries management moves to count all fish
caught against quota (the `landings obligation'), the challenge is to catch
available quota within new constraints, else lose productivity.  
%% More detailed background
A mechanism for decoupling exploitation of species caught together is spatial
targeting, but this remains challenging due to complex fishery and population
dynamics in space and time\cite{Branch2008, Dunn2014a}. 
%% General problem
How far spatial targeting can go to practically separate species is often
unknown and anecdotal.
%% Summarising the main result
Here we develop a dimension-reduction framework based on joint species
distribution modelling (spatial dynamic factor analysis) to understand how
spatial community and fishery dynamics interact to determine species and size
composition.  
%%	 Two or three sentences explaining what the main result
In the example application to the highly mixed fisheries of the Celtic Sea,
clear common spatial patterns emerge for three distinct species-groups and,
while distribution varies inter-annually, the same species-groups are
consistently found in higher densities together, with more subtle differences
within species-groups - where spatial separation may not be practically
possible. \deleted{The results highlight both opportunities for and limitations
	of the ability to spatiotemporally separate catches.}
%%  general context
We highlight the importance of dimension reduction techniques to focus
management discussion on axes of maximal separation in space and time. We
propose that spatiotemporal modelling of available data is a scientific
necessity to address the pervasive and nuanced challenges of managing mixed
fisheries.  
\end{abstract}

%%%%%%%%%%%%%%%%%%%%%%%%%%%%%%%%%%%%%%%%%
\section*{}
\subsection{Mixed fisheries and the EU landings obligation} 

\deleted{Recent} Efforts to reduce exploitation rates in commercial fisheries
have begun the process of rebuilding depleted fish populations\cite{Worm2009}.
Improved fisheries management\deleted{has the potential to} can increase
population sizes and allow increased sustainable catches, yet fisheries catch
globally remains stagnant\cite{FAO2016}. With future increased demand for fish
protein\deleted{\cite{B??n??2016}} there is an important role for well managed
fisheries in supporting future food
security\cite{Mcclanahan2015}\deleted[id=coilin]{and}\deleted{there}
\replaced[id=coilin]{necessitating }{remains a need to ensure}fisheries are
managed efficiently to maximise productivity.

A\deleted[id = brutus]{particular} challenge in realising increased catches
from rebuilt populations is maximising yields from mixed
fisheries\cite{Branch2008, Kuriyama2016, Ulrich2016}. In mixed
fisheries\deleted{, the predominant type of fishery worldwide, several fish
	species are caught together in the same net or fishing operation
	\deleted{(known as a `technical inteaction')}. If} managed by
individual quotas, \replaced[id = brutus]{if}{and} catches do not match
available stock quotas, either a vessel must stop fishing when the first quota
is reached (the `choke' species) or overexploitation of the weaker species
occurs while fishers \deleted{continue to} catch more healthy species and throw
back (`discard') the fish for which they have no quota\cite{Batsleer2015}.
There is\deleted{, therefore,} a pressing need for scientific tools which
simplify the complexities of mixed fisheries to help \replaced[id =
coilin]{avoid discarding}{managers and fishers maximise catches}. 

Sustainability of European fisheries has been hampered by this `mixed fishery
problem' for decades with large-scale discarding\cite{Uhlmann2014}.\deleted{A
	paradigm shift is being introduced} Under the EU Common Fisheries
Policy (CFP) reform of 2012\deleted{through two significant management changes.
	First}, by 2019 all fish that are caught are due to be counted against
the respective stock quota\deleted{; second, by 2020 all fish stocks must be
	fished so as to be able to produce their Maximum Sustainable Yield
	(MSY)\cite{EuropeanParliamentandCounciloftheEuropeanUnion2013}. The
	changes are}\deleted[id = coilin]{, expected to contribute to
	attainment of the goal of Good Environmental Status (GES) under the
	European Marine Strategy Framework Directive
	(MSFD;\cite{EuropeanParliament2008}) and move Europe towards an
	ecosystem based approach to fisheries management\cite{Garcia2003}}.
\deleted{Societal objectives for fisheries to achieve MSY across ecosystem
	components are paralleled by}\deleted{Individual fishers goals
	%\added[id = coilin]{are to} maximise utility; whether that be profit,
	%income or the
continuance of traditional practices.}\deleted{Under the new policy,}Unless
fishers can avoid catch of unwanted species they will have to stop fishing when
reaching their first restrictive quota,\deleted{. This} introducing a
\deleted{potential} significant cost to\deleted{fishers of} under-utilised
quota\cite{Ulrich2016} and\deleted{provides} strong incentive to mitigate such
losses\cite{Condie2013}. 

The ability\deleted{of fishers} to align \deleted{their}catch with available
quota depends on being able to exploit target species while avoiding unwanted
catch,\deleted{ Methods by which fishers can alter their fishing patterns
	include} either by switching fishing method\deleted{(e.g.  trawling to
	netting)}, changing technical gear characteristics\deleted{(e.g.
	introducing escapement panels in nets)}, or the timing and location of
fishing activity\cite{vanPutten2012a}. \deleted{For example, otter trawl gears
	are known to have higher catch rates of roundfish due to the higher
	headline and wider sweeps which herd demersal fish into the net while
	beam trawls employ chain mesh to
	%\replaced[id = coilin]{lift}{'dig'} benthic flatfish species from the
	seabed\cite{Fraser2008}.}Spatiotemporal\deleted{management}
measures\deleted{(such as time-limited fishery closures)} have been applied to
reduce unwanted catch with varying degrees of success\cite{Needle2011,
	Dunn2014a}, \added{partly because}\deleted{However, such measures}
\added{they} have \deleted{generally}been targeted at individual species
without considering associations\deleted{and interactions} among several
species. Highly mixed fisheries are complex with spatial, technological and
community interactions\deleted{combining}; \deleted{The design of
	spatio-temporal management measures which aim to allow exploitation of
	high quota stocks while protecting low quota stocks requires
	understanding of these interactions at a scale meaningful to managers
	and fishers.}\deleted{Here,} our goal is to develop a framework for
understanding these complexities. We do so by implementing a spatio-temporal
dimension reduction method and use \deleted{the} results to draw inference on
the fishery-community dynamics, creating a framework to identify trends common
among species \deleted{. We use this to} and describe \replaced[id =
coilin]{the potential for and limitations of}{where} spatial measures
\replaced[id = coilin]{to}{can contribute to} mitigate\deleted{ing} unwanted
catches in highly mixed fisheries.


\subsection{Framework for analysing spatio-temporal mixed fisheries
	interactions}

\deleted{We present a framework for analysing how far spatio-temporal avoidance
	can contribute towards mitigating imbalances in quota in mixed
	fisheries.} We \deleted{use fisheries-independent survey data to}
characterise the spatiotemporal dynamics of key components of a fish community
by \deleted{employing a geostatistical Vector Autoregressive Spatiotemporal
	model (VAST). We}implement\added{ing} a factor analysis decomposition
to describe trends in spatiotemporal dynamics of the different species as a
function of latent variables\cite{Thorson2015} representing spatial variation
(9 factors; \deleted{which we call} 'average' spatial variation) and
spatio-temporal variation (9 factors) for encounter probability and positive
catch rates (\deleted{which we call} 'positive density')
separately\cite{Thorson2015a}. \added{This allows us to} \deleted{By describing
	the species dynamics through underlying contributory spatiotemporal
	factors we can} take account of how the factors contribute to affect
catches of the species in mixed fisheries.\deleted{We use} Gaussian Markov
Random Fields (GMRFs) \deleted{to}capture spatial and temporal dependence
within and among species groups for both encounter probability and positive
density\cite{Thorson2013}.\deleted{VAST is set in a mixed modelling framework
	to allow estimation of} Fixed effects\deleted{to} account for
systematic differences driving encounter and catches\deleted{,} such as
differences in sampling efficiency (a.k.a.  catchability), while random effects
capture the spatio-temporal dynamics of the fish community.

\subsection{Dynamics of Celtic Sea fisheries}\deleted{We use the highly mixed
	demersal fisheries of the Celtic Sea as a case study.} The Celtic Sea
is a temperate sea where fisheries are spatially and temporally
complex\deleted{; mixed fisheries are undertaken by several nations using
	different gear types}\cite{Ellis2000, Gerritsen2012}. Close to 150
species have been identified in the commercial catches of the Celtic Sea, with
approximately 30 species dominating the catch\cite{Mateo2016}. We parametrise
our \deleted{spatiotemporal} model using catch data from seven
fisheries-independent surveys undertaken\deleted{in the Celtic Sea} over the
period 1990 - 2015 (Extended Data Table 1) and include nine of the main
commercial species (see Extended Data Table 2, Fig. 2) \deleted{: Atlantic cod
	(\textit{Gadus morhua}), Atlantic haddock (\textit{Melanogrammus
		aeglefinus}), Atlantic whiting (\textit{Merlangius merlangus}),
	European Hake (\textit{Merluccius merluccius}), white-bellied
	anglerfish (\textit{Lophius piscatorius}), black-bellied anglerfish
	(\textit{Lophius budegassa}), megrim (\textit{Lepidorhombus
		whiffiagonis}), European Plaice (\textit{Pleuronectes
		platessa}) and Common Sole (\textit{Solea solea}). These
	species} which make up \textgreater 60 \% of landings by towed fishing
gears for the area (average 2011 - 2015;\cite{STECF2017}). Each species was
separated into juvenile and adult size classes based on their legal minimum
conservation reference size (Extended Data Table 2).

\deleted[id = coilin]{We analyse the data to understand how the different
	associations among emergent species-groups (combination of species and
	size class) and their potential drivers affect catch compositions in
	mixed fisheries. We consider how these have changed over time, and the
	implications for mixed fisheries in managing catches of quota species
	under the EU landing obligation.}

\subsection{Common average spatial patterns driving species associations} A
spatial dynamic factor analysis decomposes the dominant spatial patterns
driving differences in encounter probability and positive density. The first
three factors\deleted{(after PCA rotation)} account for 83.7\% of the between
species variance in average encounter probability and 69\% of the between
species variance in average positive density. A clear spatial pattern can been
seen both for encounter probability and positive density, with a positive value
associated with the first factor in the inshore north easterly part of the
Celtic Sea into the Bristol Channel and Western English Channel, moving to a
negative value offshore in the south-westerly waters (Fig. 1).  The species
loadings coefficients show plaice, sole and whiting to be positively associated
with the first factor for encounter probability while the other species are
negatively associated. For average positive density, positive associations are
also found for haddock and juvenile cod. \deleted{This is indicative of a more
	inshore distribution for these species.} 

On the second spatial factor for encounter probability a north / south split
can be seen at approximately 49$^{\circ}$ N while positive density is more
driven by a positive value in the deeper westerly waters as well as some
inshore areas. Species values for the second factor indicate there are positive
associations for juvenile monkfish (\emph{L. piscatorius}), juvenile hake,
juvenile megrim, plaice and juvenile whiting with average positive density,
which may reflect two different spatial distributions in the more offshore and
in the inshore areas (Fig. 1).

On the third factor, there is a positive association with the easterly waters
for encounter probability and negative with the westerly waters. This
\deleted{manifests in the species associations as} splits\deleted{ting} the
roundfish species cod, haddock and whiting which\deleted{all} have a positive
association with the third factor for average encounter probability from the
rest of the species \deleted{which have a negative association}.  Positive
density is driven by a north / south split (Fig. 1), with positive values in
the northerly areas. Juvenile monkfish (\emph{L.  budgessa} and \emph{L.
	piscatorius}), cod, juvenile haddock, hake, adult plaice and whiting
are also positively associated with the third factor in the north while adult
monkfish (\emph{L. budgessa} and \emph{L.  piscatorius}), adult haddock,
megrims, juvenile plaice and sole are negatively associated \deleted{reflecting
	their more southerly distribution} (Fig. 1).

\deleted{While this exploratory factor analysis is modelling unobserved drivers
	of distribution,} \deleted[id = brutus]{We considered what might be
	driving the differences seen in the spatial factor loadings. The first
	factor was highly correlated with log(depth) for both encounter
	probability (-0.85, CI = -0.88 to -0.81; Extended Data Fig. 1) and
	positive density (-0.71, CI = -0.77 to -0.65; Extended Data Fig. 2),}
\deleted{A random forest classification tree assigned} \deleted[id =
brutus]{with 80 \% of the variance in the first factor for encounter
	probability to depth and predominant substrate type, with the majority
	(86 \%) of the variance explained by depth (random forest
	classification tree).} \deleted{The variance explained by these
	variables dropped to 25 \% on the second factor with a more even split
	between depth and substrate, while explaining 60 \% of the variance on
	the third factor.  For positive density, the variables explained less
	of the variance with 62 \%, 35 \%, and 31 \% for each of the factors,
	respectively.}

\deleted[id = brutus]{It is clear that depth and to a lesser extent substrate
	are important predictors for the main driver of similarities and
	differences in distributions and abundances for the different species.}
\deleted{The first factor correlates strongly with these variables, despite
	them not explicitly being incorporated in the model. While depth was
	incorporated as a covariate in an alternative model formulation (see
	Methods), it was found not to improve predictions. The utility of these
	variables as predictors of species distributions has been}\deleted[id =
brutus]{, as identified in other marine species distribution
	models\cite{Robinson2011}; the advantage to the approach taken here is
	that, where such data is unavailable at appropriate spatial resolution,
	the spatial factor analysis can adequately characterise these
	influences on species spatial dynamics.}

\subsection{\replaced[id = Paul]{Time varying species distributions, but
		stability within species groups}{Changes in spatial patterns
		over time, but stability in species dynamics}}\deleted{While
	there are clear spatial patterns in the factor coefficients describing
	differences in average (over time) encounter probability and positive
	density (Fig. 1),}\deleted{The} Inter-annual differences in factor
coefficients show less structure (Extended Data Figs. 5,6). These inter-annual
differences are important as they reflect the ability of fishers to predict
where they can target species from one year to the next.\deleted{, without
	which it may be difficult to avoid unwanted catch}\deleted{There were,
	however, While spatio-temporal factor coefficients did not show
	consistent trends from year to year across all species,} Common
patterns in spatiotemporal factor coefficients among species\deleted{there were
	clear relationships (Fig. 2)}\deleted{.  The same factors appear to}
drive spatiotemporal distributions of megrim, anglerfish species and hake (the
deeper water species, species-group negatively associated with the second axes
of Fig. 2a) and the roundfish and flatfish (species-group more positively
associated with the second axes of Fig. 2a).  For spatio-temporal positive
density (Fig. 2b) cod, haddock and whiting (the roundfish species) are
separated from plaice, sole (the flatfish) and deeper water species.
\deleted{As such,} \deleted[id = brutus]{From this it can be predicted that
	higher catches of a species within a group (e.g.  cod in roundfish)
	would be expected when catching another species within that group (e.g.
	whiting in roundfish), \deleted{. This} suggesting \deleted{that} one
	or more common environmental drivers are influencing the distributions
	of the species groups, and that driver differentially affects the
	species groups, but this could not be explained by temperature}
\deleted{is often included as a covariate in species distribution models, but
	was found not to contribute to the variance in the first factor
	values}\deleted[id = brutus]{ (Extended Data Fig. 6, no correlations
	found for either encounter probability or positive density).}

\subsection{Three clusters of species show similar spatial patterns}
\replaced[id = coilin]{To}{In order to} gain greater insight into the community
dynamics we considered how species covary in space and time through among
species correlations. Pearson correlation coefficients for the modelled average
spatial encounter probability (Fig. 3a) show clear strong associations between
adult and juvenile size classes for all species (\textgreater 0.75 for all
species except hake, 0.56).\deleted{Among species-groups,} Hierarchical
clustering identified the same three common groups\deleted[id = brutus]{as our
	visual inspection of factor loadings above,} with roundfish (cod,
haddock, whiting)\deleted{closely grouped in their association, with
	correlations for adult cod with adult haddock and adult whiting of 0.73
	and 0.5 respectively, while adult haddock with adult whiting was 0.63
	(Fig. 3a).}, flatfish (plaice and sole)\deleted{are also strongly
	correlated with adult plaice and sole having a coefficient of 0.75.
	The final group are principally the} and species found in the deeper
waters (hake, megrim and both anglerfish species)\deleted{with the megrim
	strongly associated with the budegassa anglerfish species (0.88).
	Negative relationships were found between plaice, sole and the monkfish
	species (-0.27, -0.26 for the adult size class with budegassa adults
	respectively) and hake (-0.33, -0.37) (Fig. 3a)} showing strong
intra-group correlations indicating similar spatial distributions.\deleted{This
	confirms the associations among species seen in the factor loadings,
	with three distinct species-group assemblages being
	present.}\deleted{This is also evident in} Correlation coefficients for
the average positive density also have strong associations among\deleted{the}
roundfish (Fig. 3b). \deleted{show fewer significant positive or negative
	relationships among species than for encounter probability, but still
	evident are the strong association among the roundfish with higher
	catches of cod are associated with higher catches of haddock (0.58) and
	whiting (0.47), as well as the two anglerfish species (0.71 for
	piscatorius and 0.44 for budegassa) and hake (0.73). Similarly, plaice
	and sole are closely associated (0.31) and higher catches of one would
	expect to see higher catches of the other, but also higher catches of
	some juvenile size classes of roundfish (whiting and haddock) and
	anglerfish species.  Negative association of juvenile megrim,
	anglerfish (budegassa) and hake with adult sole (-0.61, -0.61 and -0.47
	respectively), plaice (-0.36 and -0.35 for megrim and hake only)
	indicate high catches of one can predict low catches of the other
	successfully.}

\deleted{In addition to the average spatial correlations, we also estimate
	spatiotemporal correlations. Spatial population correlations
	(representing the average correlations between pairs for species
	\textit{x} and species \textit{y} across all years) are linearly
	associated with the spatiotemporal population correlations
	(representing how correlations between species \textit{x} and species
	\textit{y} change from year to year), indicating generally predictable
	relationships between species from one year to the next. This suggests
	that a positive or negative association between two species is likely
	be persist from one year to the next, and that species are consistently
	associated with each other in the catch. The correlation coefficients
	were 0.59 (0.52 - 0.66) and 0.47 (0.38 - 0.55) for encounter
	probability and positive density respectively.  However, a linear
	regression between the spatial correlations and the spatio-temporal
	correlations shows high variance (R\textsuperscript{2} = 0.36 and 0.22
	respectively), indicating that the scale of these relationships does
	change from one-year to the next. This would have implications for the
	predictability of the relationship between catches of one species and
	another when trying to balance catch with quotas in mixed fisheries. It
	can also be seen in the spatial factor maps that there are subtle
	differences in spatial patterns in factor loading values from one year
	to the next (Extended Data Figs. 4 and 5) indicating changes may be
	driven by temporally changing environmental factors and species
	behaviour.}


\subsection{Subtle differences in distributions may be important to separate
	catches within groups \deleted[id = coilin]{under the landing
		obligation}} \deleted[id = coilin]{The analysis shows the
	interdependence within the species-groups of roundfish, flatfish and
	deeper water species, where catching one species within the group
	indicates a high probability of catching the other species, which has
	important implications for how spatial avoidance can be used to support
	implementation of the EU's landings obligation. }If production from
mixed fisheries is to be maximised, decoupling catches of species between and
within the groups will be key. For example, asking where the maximal separation
in the densities of two coupled species is likely to occur?\deleted{To address
	this requirement,} We map the difference in spatial distribution
\deleted{within a group}for each pair of species within a species-group for a
single year (2015; Fig. 4) \deleted{. This would}\added{to} facilitate
discussion on maximal separation, for example, between difficult to separate
species such as haddock and whiting (Fig. 4c). \deleted{Cod had a more
	north-westerly distribution than haddock, while cod was more westerly
	distributed than whiting roughly delineated by the 7$^{\circ}$ W line
	(Fig. 4a). Whiting appeared particularly concentrated in an area
	between 51 and 52 $^{\circ}$ N and 5 and 7 $^{\circ}$ W, which can be
	seen by comparing the whiting distribution with both cod (Fig. 4b) and
	haddock (Fig. 4c). For the deeper water species (Figs. 4d and 4e), hake
	are more densely distributed in two areas compared to anglerfishes and
	%\footnote{two species combined as they are managed as one and megrim
	(though megrim has a stable density across the modelled area as
	indicated by the large amount of white space). For anglerfishes and
	megrim (Fig. 4f), anglerfishes have a more easterly distribution than
	megrim.  For the flatfish species plaice and sole (Fig. 4g), plaice
	appear to be more densely distributed along the coastal areas of
	Ireland and Britain, while sole are more densely distributed in the
	Southern part of the English Channel along the coast of France.}

Predicted catch distribution from a ``typical'' otter trawl gear and beam trawl
fishing at three different locations highlights the differences fishing gear
and location makes on catches (Fig. 4h). \deleted{As can be seen, both the gear
	selectivity and area fished play important contributions to the catch
	compositions;}In the inshore area (location `A') plaice and sole are
the two main species\deleted{in} caught reflecting their distribution and
abundance, though the otter trawl gear catches a greater proportion of plaice
to sole than the beam trawl. The area between Britain and Ireland (location
`B') has a greater contribution of whiting, haddock, cod, hake and anglerfishes
in the catch with the otter trawl catching a greater proportion of the
roundfish, haddock, whiting and cod while the beam trawl catches more
anglerfishes and megrims. The offshore area has a higher contribution of
megrim, anglerfishes and hake with the otter trawl catching a greater share of
hake and the beam trawl a greater proportion of megrim. Megrim dominates the
catch for both gears in location `C', reflecting its relative abundance in the
area.  

\subsection{Addressing the scientific challenges of the landing obligation in
	mixed fisheries} \deleted{In application to the Celtic Sea} We have
identified spatial separation of three distinct species-groups (roundfish,
flatfish and deeper water species) while showing that only subtle differences
exist in distributions within species-groups. The differences in catch
compositions between gears at the same location (Fig. 4h) show that changing
fishing methods can go some way to affecting catch, yet that differences in
catches between locations are likely to be more important. \deleted[id =
brutus]{For example, beam trawls fishing at the inshore locations (e.g.
	location `A' in Fig. 4) are likely to predominately catch plaice and
	sole, yet switching to the offshore locations (e.g. location `C') would
	likely yield greater catches of megrim and anglerfishes.
	Such}\added{This highlights that} changes in spatial fishing patterns
are likely to play an important role in supporting implementation of the
landings obligation.

More challenging is within-group spatial separation due to
\deleted{significant}overlaps in spatial distributions for the species, driven
by common environmental factors. Subtle changes in location fished may yield
some benefit in changing catch composition, yet the outcome is likely to be
much more difficult to predict.\deleted{For example,} Subtle differences in the
distribution of cod, haddock and whiting can be seen in Figs. 4a-c, showing
spatial separation of catches is \deleted[id = brutus]{much more challenging
	and}likely to need to be supported by other measures such as changes to
the selectivity characteristics of gear\cite{Santos2016}. 

A role that science can play in supporting effectiveness of spatiotemporal
avoidance could be in providing probabilistic advice on \deleted{likely}
hotspots for species\deleted{occurrence and high species density} which can
inform fishing decisions. Previous modelling studies have shown how
spatiotemporal models could improve predictions of high ratios of bycatch
species to target species\cite{Ward2015, Cosandey-Godin2015, Breivik2016}, and
geostatistical models are well suited\deleted{to this} as they incorporate
spatial dependency while providing for probabilities to be drawn from posterior
distributions of the parameter estimates. We posit\deleted{that} such advice
could be enhanced by integrating data obtained directly from commercial fishing
vessels at a higher temporal resolution, providing real-time forecasts to
inform fishing choices that also captures seasonal differences in
distributions\deleted{, akin to weather forecasting}. \added{Such}
advice\deleted{informed by a model including a seasonal or real-time component}
could inform optimal policies for time-area closures, move-on rules or even as
informal information\deleted{to be} utilised by fishers directly without being
reliant on costly continuous data collection on environmental
parameters\deleted{, but by using the vessels-as-laboratories approach}.

An important question for the implementation of the EU's landing obligation is
how far spatial avoidance can go to achieve catch balancing in fisheries. Our
model captures differences between location fished for two gear types and
\deleted{their} broad scale effect on catch composition\deleted{, information
	crucial for managers in implementing the landing
	obligation}.\deleted{Results of} Empirical studies\deleted{undertaken
	elsewhere}\cite{Branch2008, Kuriyama2016} suggest limits to the
effectiveness of spatial avoidance. Differences in ability to change catch
composition have also been observed for different fleets\deleted{; in the North
	Sea targeting ability was found to differ between otter and beam
	trawlers as well as between vessels of different
	sizes}\cite{Pascoe2007}.\deleted{It is likely}\deleted{, however, that}
\added{This} analysis \added{likely} reflects a lower bound on \deleted[id =
coilin]{the utility of spatial} avoidance as fine-scale behavioural decisions
such as time-of-day, gear configuration and location choices can also be used
to affect catch\cite{Abbott2015, Thorson2016}.

\deleted{Our framework allows for a quantitative understanding of the broad
	scale global production set available to fishers\cite{Reimer2017} and
	thus the extent to which they can alter catch compositions while
	operating in a mixed fishery.  Simulations of spatial effort allocation
	scenarios based on the production sets derived from the model estimates
	could be used as inputs to fisher behavioural models to allow for
	identification of the lower bounds of optimum spatial harvest
	strategies. This would provide managers with information useful for
	examining trade-offs in quota setting by integrating potential for
	spatial targeting in changing catch composition, thus provide a
	scientific contribution to meeting the goal of maximising catches in
	mixed fisheries within single stock quota constraints\cite{Ulrich2016}.
	Further, the correlations among species could provide information on
	fisheries at risk of capturing protected, endangered or threatened
	species such as elasmobranches, and allow identification of areas where
	there are high ratios of protected to target species.}


Complex environmental, fishery and community drivers of distribution for
\deleted[id = brutus]{groups of}species highlights the scale of the challenge
in separating catches\deleted[id = brutus]{within the species-groups} using
spatial management measures. This has important implications for management of
the mixed fisheries under the EU landings obligation. Our analysis identifies
where it may be easier to separate catches of species (among groups) and where
it is more challenging (within groups). We propose that the framework presented
in Figs. 1-4 provides a viable route to reducing the complexity of highly mixed
systems.  This can allow informed management discussion over more traditional
anecdotal knowledge of single-species distribution in space and time.


\section*{Methods}

\subsection{Model structure:} \added{We use a geostatistical Vector
	Autoregressive Spatiotemporal model (VAST)}\footnote{Software in the R
	statistical programming language can be found here:
	\url{www.github.com/james-thorson/VAST}} to implement a
delta-generalised linear mixed modelling (GLMM) framework that takes account of
spatio-temporal correlations among species through implementation of a spatial
dynamic factor analysis (SDFA). Spatial variation is captured through a
Gaussian Markov Random Field, while we model random variation among species and
years. Covariates affecting catchability (to account for differences between
fishing surveys) and density (to account for environmental preferences) can be
incorporated for predictions of presence and positive density. The following
briefly summarises the key methods implemented in the VAST framework. For full
details of the model the reader is invited directed to Thorson \textit{et al}
2017\cite{Thorson2017}.

\textbf{\textit{SDFA:}} A spatial dynamic factor analysis incorporates advances
in joint dynamic species models\cite{Thorson2017} to take account of
associations among species by modelling response variables as a multivariate
process. This is achieved through implementing a factor analysis decomposition
where common latent trends are estimated so that the number of common trends is
less than the number of species modelled. The factor coefficients are then
associated through a function for each factor that returns a positive or
negative association of one or more species with any location. Log-density of
any species is then be described as a linear combination of factors and
loadings: \begin{equation} \theta_{c}(s,t) = \sum_{j=1}^{n_{j}}
	L_{c,j}\psi_{j}(s,t) +\sum_{k=1}^{n_{k}} \gamma_{k,c}\chi_{k}(s,t)
\end{equation} Where $\theta_{c}(s,t)$ represents log-density for species $c$
at site $s$ at time $t$, $\psi_{j}$ is the coefficient for factor $j$,
$L_{c,j}$ the loading matrix representing association of species $c$ with
factor $j$ and $\gamma_{k,c}\chi_{k}(s,t)$ the linear effect of covariates at
each site and time\cite{Thorson2016b}. 

The factor analysis can identify community dynamics and where species have
similar spatio-temporal patterns, allowing inference of species distributions
and abundance of poorly sampled species through association with other species
and allows for computation of spatio-temporal correlations among
species\cite{Thorson2016b}.

\added[id = brutus]{We use the resultant factor analysis is used to identify
	community dynamics and drivers common among 18 species and results
	presented through transformation of the loading matrices using PCA
	rotation}. 

\textbf{\textit{Estimation of abundances:}} Spatio-temporal encounter
probability and positive catch rates are modelled separately with
spatio-temporal encounter probability modelled using a logit-link linear
predictor;
		\begin{equation}
			\begin{split}
			logit[p(s_{i},c_{i},t_{i})] =	\beta_{p}(c_{i},t_{i}) +
			& \sum\limits_{f=1}^{n_{\omega}} L_{\omega}(c_{i},f)
			\omega_{p}(s_{i},f) + \sum\limits_{f=1}^{n_{\varepsilon}}
			L_{\varepsilon}(c_{i},f) \varepsilon_{p}(s_{i},f,t_{i}) + \\ 
			& \sum\limits_{v=1}^{n_{v}}\delta_{p}(v)Q_{p}(c_{i}, v_{i})
		\end{split}
		\end{equation}

and positive catch rates modelling using a gamma- distribution\cite{Thorson2015a}. 
		\begin{equation}
			\begin{split}
			log[r(s_{i},c_{i},t_{i})] = \beta_{r}(c_{i},t_{i}) +
			& \sum\limits_{f=1}^{n_{\omega}} L_{\omega}(c_{i},f)
			\omega_{r}(s_{i},f) +\sum\limits_{f=1}^{n_{\varepsilon}} 
			L_{\varepsilon}(c_{i},f) \varepsilon_{r}(s_{i},f,t_{i}) + \\
			& \sum\limits_{v=1}^{n_{v}}\delta_{r}(v) Q_{r}(c_{i}, v_{i})
			\end{split}
		\end{equation}

where $p(s_{i}, c_{i}, t_{i})$ is the predictor for encounter probability for
observation $i$, at location $s$ for species $c$ and time $t$ and $r(s_{i},
c_{i}, t_{i})$ is similarly the predictor for the positive density.
$\beta_{*}(c_{i},t_{i})$ is the intercept, $\omega_{*}(s_{i},c_{i})$ the
spatial variation at location $s$ for factor $f$, with $L_{\omega}(c_{i},f)$
the loading matrix for spatial covariation among species.
$\varepsilon_{*}(s_{i},c_{i},t_{i})$ is the linear predictor for
spatio-temporal variation, with $L_{\varepsilon}(c_{i}, f)$ the loading matrix
for spatio-temporal covariance among species and $\delta_{*}(c_{i}, v_{i})$ the
contribution of catchability covariates for the linear predictor with
$Q_{c_{i}, v_{i}}$ the catchability covariates for species $c$ and vessel $v$;
$*$ can be either $p$ for probability of encounter or $r$ for positive density.

The Delta-Gamma formulation is then:
\begin{equation}
	\begin{split}
	& Pr(C = 0) = 1 - p \\
	& Pr(C = c | c > 0) = p \cdot \frac{\lambda^{k}c^{k-1} \cdot exp(-\lambda c)}{\Gamma_{k}}
	\end{split}
\end{equation}

for the probability $p$ of a non-zero catch $C$ given a gamma distribution for
for the positive catch with a rate parameter $\lambda$ and shape parameter $k$.

\textbf{\textit{Spatio-temporal variation:}} The spatiotemporal variation is
modelled using Gaussian Markov Random Fields (GMRF) where data is associated to
nearby locations through a Matérn covariance function with the parameters
estimated within the model. Here, the correlation decays smoothly over space
the further from the location and includes geometric anisotropy to reflect the
fact that correlations may decline in one direction faster than another (e.g.
moving offshore)\cite{Thorson2013}.  The best fit estimated an anisotropic
covariance where the correlations were stronger in a north-east - south-west
direction, extending approximately 97 km and 140 km before correlations for
encounter probability and positive density reduced to \textless 10 \%,
respectively (Extended Data Fig. 9). Incorporating the spatiotemporal
correlations among and within species provides more efficient use of the data
as inference can be made about poorly sampled locations from the covariance
structure.

A probability distribution for spatio-temporal variation in both encounter
probability and positive catch rate was specified, $\varepsilon_{*}(s,p,t)$,
with a three-dimensional multivariate normal distribution so that:
	\begin{equation}
		vec[\mathbf{E}_{*}(t)] \sim MVN(0,\mathbf{R}_{*} \otimes
		\mathbf{V}_{{\varepsilon}{*}})
	\end{equation}

Here, $vec[\mathbf{E}_{*}(t)]$ is the stacked columns of the matrices
describing $\varepsilon{*}(s,p,t)$ at every location, species and time,
$\mathbf{R}_{*}$ is a correlation matrix for encounter probability or positive
catch rates among locations and $\mathbf{V}_{*}$ a covariance matrix for
encounter probability or positive catch rate among species (modelled within the
factor analysis). $\otimes$ represents the Kronecker product so that the
correlation among any location and species can be computed\cite{Thorson2017}.
		
\textbf{\textit{Incorporating covariates}} Survey catchability (the relative
efficiency of a gear catching a species) was estimated as a fixed effect in the
model, $\delta_{s}(v)$, to account for differences in spatial fishing patterns
and gear characteristics which affect encounter and capture probability of the
sampling gear\cite{Thorson2014}. Parameter estimates (Extended Data Fig. 10)
showed clear differential effects of surveys using otter trawl gears (more
effective for round fish species) and beam trawl gears (more effective for
flatfish species).

No fixed covariates for habitat quality or other predictors of encounter
probability or positive density were included. While incorporation may improve
the spatial predictive performance\cite{Thorson2017}, it was not found to be
the case here based on model selection with Akaike Information Criterion (AIC)
and Bayesian Information Criterion (BIC).

\textbf{\textit{Parameter estimation}} Parameter estimation was undertaken
through Laplace approximation of the marginal likelihood for fixed effects
while integrating the joint likelihood (which includes the probability of the
random effects) with respect to random effects. This was implemented using
Template Model Builder (TMB;\cite{Kristensen2015}) with computation through
support by the Irish Centre for High End Computing (ICHEC;
\url{https://www.ichec.ie}) facility.  

\subsection{Data}

The model integrates data from seven fisheries independent surveys taking
account of correlations among species spatio-temporal distributions and
abundances to predict spatial density estimates consistent with the resolution
of the data. 

The model was been fit to nine species separated into adult and juvenile size
classes (Extended Data Table 2) to seven survey series (Extended Data Table 1)
in the Celtic Sea bound by 48$^{\circ}$ N to 52 $^{\circ}$ N latitude and 12
$^{\circ}$ W to 2$^{\circ}$ W longitude (Extended Data Fig. 8) for the years
1990 - 2015 inclusive. 

The following steps were undertaken for data processing: i) data for survey
stations and catches were downloaded from ICES Datras
(\url{www.ices.dk/marine-data/data-portals/Pages/DATRAS.aspx}) or obtained
directly from the Cefas Fishing Survey System (FSS); ii) data were checked and
any tows with missing or erroneously recorded station information (e.g. tow
duration or distance infeasible) removed; iii) swept area for each of the
survey tows was estimated based on fitting a GAM to gear variables so that
Doorspread = s(Depth) + DoorWt + WarpLength + WarpDiameter + SweepLength and a
gear specific correction factor taken from the literature\cite{Piet2009}; iii)
fish lengths were converted to biomass (Kg) through estimating a von
bertalanffy length weight relationship, $Wt = a \cdot L^{b}$, fit to sampled
length and weight of fish obtained in the EVHOE survey and aggregated within
size classes (adult and juvenile). 

The final dataset comprised of estimates of catches (including zeros) for each
station and species and estimated swept area for the tow.

\subsection{Model setup}

The spatial domain was setup to include 250 knots representing the Gaussian
Random Fields. The model was configured to estimate nine factors each to describe
the spatial and spatiotemporal encounter probability and positive density
parameters, with a logit-link for the linear predictor for encounter
probability and log-link for the linear predictor for positive density, with an
assumed gamma distribution.

Three candidate models were identified, i) a base model where the vessel
interaction was a random effect, ii) the base but where the vessel x species
effect was estimated as a fixed covariate, iii) with vessel x species effect
estimated, but with the addition of estimating fixed density covariates for
both predominant habitat type at a knot and depth. AIC and BIC model selection
favoured the second model (Extended Data Table 3). The final model included
estimating 130,950 coefficients (1,674 fixed parameters and 129,276 random
effect values).

\subsection{Model validation}

Q-Q plots show good fit between the derived estimates and the data for positive
catch rates and between the predicted and observed encounter probability
(Extended Data Figs. 11,12).  Further, model outputs are consistent with
stock-level trends abundances over time from international assessments
(Extended Data Fig. 13), yet also provide detailed insight into species
co-occurrence and the strength of associations in space and time. 

%%%%%%%%%%%%%%%%%%%%%%%%%%%%%%%%%%%%%%%%%%%%%%%%%%%%%%%%%%%%%%%%%
\end{linenumbers}
\newpage
\Urlmuskip=0mu plus 1mu\relax
\bibliographystyle{naturemag}
\small{\bibliography{../JSDM}}

%%%%%%%%%%%%%%%%%%%%%%%%%%%%%%%%%%%%%%%%%%%%%%%%%%%%%%%%%%%%%%%%%%%

\newpage

%% Here is the endmatter stuff: Supplementary Info, etc.
%% Use \item's to separate, default label is "Acknowledgements"

\begin{addendum}
 \item [Acknowledgements] Paul J Dolder gratefully acknowledges funding support
	 from the MARES joint doctoral research programme (MARES\_14\_15) and
	 Cefas seedcorn (DP227AC) and logistical support, desk space and
	 enlightening discussions with Trevor Branch, Peter Kuriyama, Cole
	 Monnahan and John Trochta at the School of Aquatic and Fisheries
	 Science (SAFS) at the University of Washington during a study visit.
	 
	 The authors gratefully acknowledge the hard-work of many scientists
	 and crew in collecting and storing data during the numerous scientific
	 surveys used in this study without which it would not have been
	 possible.  
	 
	 The manuscript benefited greatly from discussions with David Stokes,
	 Colm Lordan, Claire Moore and Hans Gerritsen (Marine Institute,
	 Ireland), Lisa Readdy, Chris Darby, Ian Holmes, Stephen Shaw and Tim
	 Earl (Cefas).  The authors are very grateful to Lisa Readdy for
	 provision of the Cefas datasets.

 \item[Author contributions] P.J.D., C.M and J.T.T. designed the study. P.J.D.
	 conducted the analysis. All authors contributed to writing the
	 manuscript.  

 \item[Competing Interests] The authors declare that they have
	 no competing financial interests.
 \item[Correspondence] Correspondence and requests for materials
 should be addressed to Paul Dolder (email: paul.dolder@gmit.ie).
 \end{addendum}

%%%%%%%%%%%%%%%%%%%%%%%%%%%%%%%%%%%%%%%%%%%%%%%%%%%%%%%%%%%%%%%%%%%

%%%

%% Figure 1 - spatial factors
\begin{figure}
\begin{center}
	\includegraphics[width=\linewidth]{"figures/Fig1_Combined"}
	\label{fig:1}
	\caption{Factor values for the first three factors for (a) Average
		encounter probability and (b) Average positive density for the
		species (outer figures) and spatially (inner figures).
		Red: positive association to the factor, Blue: negative
		association}
\end{center}
\end{figure}


%% Figure 2 - PCA style plots
\begin{figure}
\begin{center}
	\includegraphics[width=\linewidth]{"figures/Figure 3 - PCAstyle_Plots_SpatioTemp"}
	\label{fig:2}
	\caption{Position of each species on the first two axes from the
		factor analysis for (a) spatio-temporal encounter probability
		and (b) spatio-temporal positive density.}
\end{center}
\end{figure}

\begin{figure}
\begin{center}
	\includegraphics[width = \linewidth]{"figures/Figure 1 - Omega1Omega2_Correlations_blank"}
	\label{fig:3}
	\caption{Inter-species correlations for (a) spatial encounter
		probability over all years and (b) spatial positive density.
		Species are clustered into three groups based on a
		hierarchical clustering method with non-significant
		correlations (the Confidence Interval [+- 1.96 * SEs] spanned
		zero) left blank.}
	\end{center}
\end{figure}

\begin{figure}
\begin{center}
	\includegraphics[width = \linewidth]{"figures/Figure 4 - DensityDifferencesFigureswithCC"}
	\label{fig:4}
	\caption{Differences in the standardised spatial density for pairs of
		species and expected catch rates for two different gears at
		three different locations in 2015.}
\end{center}
\end{figure}


\end{document}


